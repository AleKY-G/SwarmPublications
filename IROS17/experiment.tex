
%%%%%%%%%%%%%%%%%%%%%%%%%%%%%%%%%%%%%%%%%%%%%%%%%%%%%%%%%%%
\section{experiment}\label{sec:expResults}
%%%%%%%%%%%%%%%%%%%%%%%%%%%%%%%%%%%%%%%%%%%%%%%%%%%%%%%%%%%

\subsection{Hardware Experiment: Position Control of n Robots}
A hardware setup with a bounded platform, magnetic sliders, and a magnetic guide board was used to implement  Alg. \ref{alg:PosControlNRobots}. 
 % Designs for each are available at \cite{Arun2016Thingiverse}. 
  The pink boundary is toothed with a white free space, as shown in Fig \ref{fig:construction2d}.
    Only discrete, 1 cm moves in the $x$ and $y$ directions are used. The goal configuration highlighted in the top right corner represents a `U' made of seven sliders. The dark red configuration is the current position of the sliders. 
Due to the discretized movements allowed by the boundary, drift moves follow a 1 cm square.  Free robots return to their start positions but robots on the boundary to move laterally, generating a net sliding motion.

Fig. \ref{fig:construction2d} follows the motion of the sliders through iterations  $k$=1, 2, and 7. All robots receive the same control inputs, but boundary interactions break the control symmetry.  Robots reach their goal positions in a first-in, first-out arrangement beginning with the bottom-left robot from the staging zone occupying the top-right position of the build zone.

