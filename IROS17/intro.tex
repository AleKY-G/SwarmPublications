\section{Introduction}\label{sec:Intro}
Particle swarms propelled by a global field, where each particle  receives the same control input, are common in applied mathematics, biology, and computer graphics. 
%If the $i$th particle has position $[x_i,y_i]$ and velocity $[\dot{x}_i, \dot{y}_i]$, then
%\begin{align}
%[\dot{x}_i, \dot{y}_i]^\top = \mathbf{u}(t), \qquad i \in [1,n]. \label{eq:swarmDynamics} %[u_x, u_y]^\top
%\end{align}
%These system dynamics represent particle swarms in low-Reynolds number environments, where viscosity dominates inertial forces and so velocity is proportional to input force~\citep{Purcell1977}. 
% In this regime, the input force command $\mathbf{u}(t)$ controls the velocity of the robots.  
%  The same model can be generalized to particles moved by fluid flow where the vector direction of fluid flow $\mathbf{u}(t)$ controls the velocity of particles, or for a swarm of robots that move at a constant speed in a direction specified by a global input $\mathbf{u}(t)$~\citep{Rubenstein2012}.
% Our control problem is to design the control inputs $\mathbf{u}(t)$ to make all $n$ particles achieve a task.
As a current example, micro- and nano-robots can be manufactured in large numbers, see~\cite{Chowdhury2015,martel2014computer,kim2015imparting,Donald2013,Ghosh2009,Ou2013,qiu2015magnetic}.
Someday large swarms of robots will be remotely guided
to assemble structures in parallel and 
 through the human body to cure disease, heal tissue, and prevent infection. %Both tasks are high   as described in \citep{munoz2014review} \citep{sitti2015biomedical} . 
 For each task, large numbers of micro robots are required to deliver sufficient payloads, but the small size of these robots makes it difficult to perform onboard computation.  Instead, these robots are often controlled by a global, broadcast signal. 
 The tiny robots themselves are often just rigid bodies, and it may be more accurate to define the \emph{system}, consisting of particles, a global control field, and sensing, as the robot.
Such systems are severely underactuated, having 2 degrees of freedom in the shared control input, but $2n$ degrees of freedom for the particle swarm.
 Techniques are needed that can handle this underactuation. 
 In previous work, we showed that the 2D position of each particle in such a swarm is controllable if the workspace contains a single obstacle the size of one particle.


Positioning is a foundational capability for a robotic system, but requiring a single, small, rigid obstacle suspended in the middle of the workspace is often an unreasonable constraint, especially in 3D.
This paper relaxes that constraint, and provides position control algorithms that only require interactions with the boundaries.
We assume that particles in contact with the boundaries have zero velocity if the global control input pushes the particle into the wall.



\begin{figure}
\centering
\vspace{1.5em}
\begin{overpic}[width=0.47\columnwidth]{story3Moves1.pdf}\put(40, 102){$n$=2}\put(20, 4){{\tiny$\updownarrow$}~$\epsilon$}\end{overpic}
\begin{overpic}[width=0.47\columnwidth]{story3Moves4.pdf}\put(10, 102){optimal shortest path}\end{overpic}\\
\vspace{1.5em}
\begin{overpic}[width=0.47\columnwidth]{monalisaBegin.pdf}\put(25, 102){$n$=862}\put(16,60){goal}\put(30,20){robots}\end{overpic}
\begin{overpic}[width=0.47\columnwidth]{monalisa.pdf}\put(2, 102){position control algorithm}\put(12,25){final configuration}\end{overpic}
\caption{\label{fig:IntroPic}
Positioning particles that receive the same control inputs, but cannot move while a control input pushes them into a boundary.
Top row shows the output of a best-first-search algorithm that finds the shortest path for two particles.
 Top left shows the initial position and goal position of the particles.  
 The shortest path consists of moving at angle 50$^\circ$ until the blue robot contacts the top wall, then moving the  magenta robot at angle 165$^\circ$ until the particles reach the desired relative spacing, then moving -60$^\circ$ to the goal positions.
 The two bottom pictures shows $n$-particle positioning using shared control inputs and boundary interaction. 
} \vspace{-1em}
\end{figure}



The paper is arranged as follows. 
After a review of recent related work in Sec.  \ref{sec:RelatedWork},
 %Sec. \ref{subsec:FluidInTank} provides analytical position control results of stable configurations in two canonical workspaces with frictionless walls.  These results are limited in the set of shapes that can be generated.  To extend the range of possible shapes,
  Sec.  \ref{subsec:WallFriction} introduces a  model for boundary interaction.  
We provide an optimal shortest-path algorithm for  arbitrarily position two robots in Sec.  \ref{sec:PostionControl2Robots}, and Sec.  \ref{sec:PostionControlnRobots} extends this to prove a rectangular workspace with high-friction boundaries can position a swarm of $n$ robots arbitrarily within a subset of the workspace.
Sec.  \ref{sec:simulation} describes implementations of the algorithms in simulation and  Sec.  \ref{sec:expResults} describes hardware experiments, as shown in Fig.~\ref{fig:IntroPic}. 
 We end with directions for future research in Sec.  \ref{sec:conclusion}.




