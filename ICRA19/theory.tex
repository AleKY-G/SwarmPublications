%%%%%%%%%%%%%%%%%%%%%%%%%%%%%%%%%%%%%%%%%%%%%%%%%%%%%%%%%%%
%\section{Torque control}
%\label{sec:theory}
%%%%%%%%%%%%%%%%%%%%%%%%%%%%%%%%%%%%%%%%%%%%%%%%%%%%%%%%%%%
%\subsection{Controlling Torque}

%We derive inspiration from recent work on pulling with a swarm \cite{pulltogether}. The contribution of this work is to map swarm distributions to torque production.
%The orientation of an object's major axis is important when a swarm is manipulating a non-symmetric object through narrow corridors. 
%Orientation is controllable by applying torque to the object. 
%To change the output torque $\tau$ in~\eqref{eq:torque}, we can choose the direction and magnitude of the force applied $F$, and the moment arm from the object's center of mass (COM) $O$ to the point of contact $P$.
%We define a coordinate frame rooted at the COM, with the $x$-axis parallel to the object's longest axis. The resulting torque is:
%
%\begin{equation}
%\tau = F \times (P - O ).\label{eq:torque}
%\end{equation}
%
%We assume that the forces produced by individual particles add linearly. 
% As \cite{pulltogether} indicates, this is often a simplification of the true dynamics. 
% The swarm version of \eqref{eq:torque} is the summation of the forces contributed by $n$ individual particles:
%\begin{align}
%\tau_{\text{total}} &= \sum\limits_{i=1}^n \rho_i F_i \times (P_i - O ) \textrm{      and}   \label{eq:swarmtorque}\\
%F_{\text{total}} &= \sum\limits_{i=1}^n \rho_i F_i.  \label{eq:swarmforce}
%\end{align}
%Here $F_i$ is the force that the $i^{\textrm{th}}$ particle applies. 
%Not all particles are in contact with the object.  
%$\rho_i$ is an indicator variable: $\rho_i=1$ if the particle is in direct contact with the object or touching a chain of particle where at least one particle is in contact with the object. 
% Otherwise $\rho_i = 0$.
%The moment arm is the particle's position $P_i$ to the object's COM $O=[O_x,O_y]^{\top}$. 
% If all particles are identical and the control input is uniform, the force is equivalent for every particle and so $F_i $ equals some constant.
%

\section{Angle of repose}\label{sec:angle}

 \begin{figure}
\centering
\renewcommand{\figwid}{\columnwidth}
\begin{overpic}[width =\figwid]{Angle.pdf}%\put(1,55){a)}
\end{overpic}
\caption{\label{fig:angle}  If particles move faster than the (pink) rod, some particles slide past the rod, but the ones that remain pile up in a shape characterized by the angle of repose $\alpha$, which is particular to the particle type.
%Different values of angle of repose is shown when granular particles move faster than the rod.
}
\end{figure}


Consider a swarm of granular particles applying force to a rod. 
If the rod moves slower than the particles, some particles will slide past the rod, but other particles will build up behind the rod in a characteristic triangular shape %defined by an apex angle. 
 defined by a steepest angle of descent perpendicular to the direction of particle motion. 
This piling up is common to all granular media, and the angle formed perpendicular to the angle of attack  is the \emph{angle of repose}. Three different values of angle of repose is shown in Fig.~\ref{fig:angle}. The center of mass of the rod is in the middle of the rod, but center of mass of the granular particles changes for different values of angle of repose. % for the media, see \cite{angleofRepose}. 
 By measuring the angle of repose for the particles shown in the top plot of Fig.~\ref{fig:AngleOfReposeForce}, we can estimate the force and torque that the swarm is applying to the rod as a function of the rod's length, the angle of repose, and orientation of the rod.
 In this plot, particulate is moving in the $-y$ direction, and the rods are tilted at $\theta=\{-45,-22.5,0,22.5,45\}^\circ$ with respect to the $x$ axis. 
  A thin black line extends upwards from the rod COM, and the COM of the particulate is shown by a white and black disk.   
  More particulate is heaped on the right side of the rod for positive $\theta$ and more on the left side for negative $\theta$. 
  This uneven particulate generates a restoring torque. 
 We define the angle of repose as $\alpha$, the rod's orientation relative to 90$^\circ$ from the particle movement vector as $\theta$, and the rod's length as $\ell$. 

\subsection{Force and Torque}

By integrating over the triangular shape, the force applied to the rod (when a unit area of particles produces 1 N of force) is
\textcolor{red}{?? TODO: Aaron INsert the integral}
\begin{figure}
\centering
\renewcommand{\figwid}{\columnwidth}
\begin{overpic}[width =\figwid]{AngleOfRepose.pdf}%\put(1,55){a)}
\end{overpic}\\
\vspace{0.5em}
\begin{overpic}[width =\figwid]{AngleOfReposeForce.pdf}%\put(1,55){a)}
\end{overpic}\\
\vspace{0.5em}
\begin{overpic}[width =\figwid]{AngleOfReposeTorque.pdf}%\put(1,55){a)}
\end{overpic}
\vspace{-0.5em}
\caption{\label{fig:AngleOfReposeForce} Top plot shows colored particulate heaped up on pink-colored long rods. 
 Middle plot shows the force applied to the rod and bottom the torque as a function of $\theta$ for four angle of repose values.
   The maximum torque values from \eqref{eq:maxTorqueAngleGivenAOR} are shown with black dots, producing a line that is approximately $-\ell^3/36 \theta_{t_{\max}}$.
% Generating code is in the attachment. 
\vspace{-2em}
}
\end{figure}

\begin{align}
F(\theta,\alpha,\ell) =\left\{
\begin{array}{ll}
\frac{\ell^2\Big(\cos(2\theta)-\cos(2\alpha)\Big)}{8\cos(\alpha)\sin(\theta)} &   -\alpha<\theta<\alpha\\
0 &    \textrm{otherwise .}\\
\end{array} 
\right . \label{eq:ForceAOR}
\end{align}


%\begin{align}
%F(\theta,\alpha,l)  = \frac{l^2}{8\cos\alpha\sin{\theta}} \Big(\cos(2\theta)-\cos(2\alpha)\Big).
%\end{align}

The force for different angle of repose values are shown in the middle plot of Fig.~\ref{fig:AngleOfReposeForce}, with the rod length $\ell=1$. 
Torque will also be similarly defined as
\begin{align}
\tau(\theta, \alpha, \ell) =\left\{
\begin{array}{ll}
\frac{\ell^3\Big(\cos(2\alpha)-\cos(2\theta)\Big)\sin(\theta) }{48\sin^2(\alpha)}&   -\alpha<\theta<\alpha\\
0 &    \textrm{otherwise .}\\
\end{array} 
\right.
\end{align}
Torque is shown in the bottom plot of Fig.~\ref{fig:AngleOfReposeForce} with the rod length $\ell=1$. 
 Given sufficient particles to pile up to the angle of repose, this torque tends to stabilize the object to be perpendicular to the pushing direction.
Force is maximized with $\theta=0$, but the $\theta$ value that maximizes torque is a function of $\alpha$ and is defined as
\begin{align}\label{eq:maxTorqueAngleGivenAOR}
\theta_{t_{\max}} = \frac{\sin(\alpha)}{\sqrt{3}}.
\end{align}
To maximize the torque a particulate swarm applies on a thin rod, the swarm should move in the direction $-\theta_{t_{\max}} - 90^\circ$ with respect to the long axis of the rod.

\subsection{Shape Control}

Often the angle of the rigid rod is given, but we can choose the desired approach direction $\beta$. 
 This section examines the possible shapes that can be generated given a rigid rod of length $\ell$, an angle of repose $\alpha$, and an approach angle where the swarm moves at angle relative to the long axis of the rod $\beta$.

Computing means, variances, covariance, and correlation requires integrating over $R$, the region containing the swarm and are calculated as%This is simplified using an indicator function $\bm{1}_A(x,y)$ that returns 1 if inside the region containing the swarm, and 0 else. 
%The formulas for means $(\bar{x},\bar{y})$, covariance $(\sigma^2_x,\sigma^2_y,\sigma_{xy})$, and correlation $\rho_{xy}$ are as follows: %, integrated over the unit square with $x$ and $y$ from 0 to 1:
%
\begin{align}
A &=\iint_R \,dx\,dy \text{, }
\bar{x} =\frac{\iint_R x \,dx\,dy}{A} \label{eq:meanInSquareWorkspace}
\text{, } \bar{y}=\frac{\iint_R y \,dx\,dy}{A} \text{, }\\
%\end{align}
%\begin{align}
\sigma^2_x &=\frac{\iint_R \left(x-\bar{x}\right)^2  \,dx \,dy}{A}  \label{eq:varInSquareWorkspace}
\text{, } \sigma^2_y =\frac{\iint_R  \left(y-\bar{y}\right)^2 \,dx \,dy}{A}  \text{, }\\
%\end{align}
%\begin{align}
\sigma_{xy} &= \frac{\iint_R  \left(x-\bar{x}\right) \left(y-\bar{y}\right) \, dx \,dy}{A} \label{eq:covAndcorrInSquareWorkspace}
\text{, }\rho_{xy} = \frac{\sigma_{xy}}{\sqrt{\sigma^2_x\sigma^2_y}}.
\end{align}

Using these equations, we can calculate the representative shape statistics as a function of approach angle $\beta$, a one degree-of-freedom set. 
For instance, the area $A$ of the particles is \eqref{eq:ForceAOR} with $\theta = \beta + \pi/2$, and the $y$-variance is
\begin{align}
\sigma_y^2(\alpha, \beta,\ell) &= \frac{1}{72}   \left( \cot(2\alpha) + \frac{\cos(2\beta)}{ \cos(2\alpha)} \right)^2.
\end{align}
Representative results are shown in Fig.~\ref{fig:AngleOfReposeStatistics}.



\begin{figure}
\centering
\renewcommand{\figwid}{\columnwidth}
\begin{overpic}[width =\figwid]{aorYmean}%\put(1,55){a)}
\end{overpic}\\
\vspace{0.5em}
\begin{overpic}[width =\figwid]{aorVarX}%\put(1,55){a)}
\end{overpic}\\
\vspace{0.5em}
\begin{overpic}[width =\figwid]{aorCovariance}%\put(1,55){a)}
\end{overpic}
\vspace{-0.5em}
\caption{\label{fig:AngleOfReposeStatistics} 
Each plot shows a statistic of the swarm shape for several angle of repose $\alpha$ values as a function of the approach angle $\beta$ with $\ell = 1$.
Top plot shows the mean $y$ coordinate $\bar{y}$,
 middle plot shows $x$-variance $\sigma^2_x$, and
 bottom plot shows the covariance $\sigma_{xy}$.
\vspace{-2em}
}
\end{figure}

