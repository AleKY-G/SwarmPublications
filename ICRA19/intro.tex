\section{Introduction}\label{sec:Intro}

  % \textcolor{red}{TODO: this text is from a previous paper}

%Particle swarms propelled by a uniform field, where each particle  receives the same control input, are common in applied mathematics, biology, and computer graphics \cite{Peyer2013,Shirai2005,Chiang2011}.
%If the $i$th particle has position $[x_i,y_i]$ and velocity $[\dot{x}_i, \dot{y}_i]$, then
%\begin{align}
%[\dot{x}_i, \dot{y}_i]^\top = \mathbf{u}(t), \qquad i \in [1,n]. \label{eq:swarmDynamics} %[u_x, u_y]^\top
%\end{align}
%These system dynamics represent particle swarms in low-Reynolds number environments, where viscosity dominates inertial forces and so velocity is proportional to input force~\citep{Purcell1977}. 
% In this regime, the input force command $\mathbf{u}(t)$ controls the velocity of the robots.  
%  The same model can be generalized to particles moved by fluid flow where the vector direction of fluid flow $\mathbf{u}(t)$ controls the velocity of particles, or for a swarm of robots that move at a constant speed in a direction specified by a global input $\mathbf{u}(t)$~\citep{Rubenstein2012}.
% Our control problem is to design the control inputs $\mathbf{u}(t)$ to make all $n$ particles achieve a task.
%As a current example, micro- and nano-robots can be manufactured in large numbers, see~\cite{Chowdhury2015,martel2014computer,kim2015imparting,Donald2013,Ghosh2009,Ou2013,qiu2015magnetic}.
%Someday large swarms of robots will be remotely guided
%to assemble structures in parallel and 
% through the human body to cure disease, heal tissue, and prevent infection. %Both tasks are high   as described in \citep{munoz2014review} \citep{sitti2015biomedical} . 
% For each task, large numbers of micro robots are required to deliver sufficient payloads, but 

This paper investigates the set of configurations that can be stably achieved for a  large group of particles, all controlled by the same global force, in a workspace with rigid obstacles.
Shape control of many particles is essential for navigation, construction, and conveying information. 
 The small size of the particles makes it hard or even impossible to control each particle's position individually. Instead, these particles are often controlled with a single global control input.
   This work analyses interactions between the rigid obstacles and the particles to control the configuration of the particles.
    % In \cite{shahrokhi2018TRO} we showed how to control swarm statistics including the mean position and variance. 
    % This work extends that work to shape control.



%The small size of these robots makes it difficult to perform onboard computation.  Instead, these robots are often controlled by a broadcast signal. 
 %The tiny robots themselves are often just rigid bodies, and it may be more accurate to define the robot as the \emph{system} that consists of particles, a uniform control field, and sensing.
%Such systems are severely underactuated, having 2 degrees of freedom in the shared planar control input, but $2n$ degrees of freedom for the $n$-particle swarm.
% Techniques are needed that can handle this underactuation. 

% Positioning is a foundational capability for a robotic system, e.g. placement of brachytherapy seeds. 
% In previous work, we provided position control algorithms that only require non-slip wall contacts to position two robots \cite{shahrokhi2017}. 
%We assumed that particles in contact with the boundaries have zero velocity if the uniform control input pushes the particle into the wall. This work analyses interactions between the boundary and the particle swarm to control shape of the swarm. In \cite{shahrokhi2018TRO} we showed how to control swarm statistics including mean position and variance. This work extends that work to covariance control.

 \begin{figure}
\centering
\renewcommand{\figwid}{\columnwidth}
\begin{overpic}[width =\figwid]{firstpic.pdf}%\put(1,55){a)}
\end{overpic}
\vspace{-2em}
\caption{\label{fig:leadfigure} 
     Reshaping particles by collisions with rigid bodies. Top pictures illustrate using closed boundaries to reshape the particles. 
      Bottom pictures illustrate with different particles how varying the the angle of attack modifies the shape of particles remaining on a rigid object and how modifying the applied force changes the angle of repose.
}
\vspace{-1em}
\end{figure}

%    \begin{figure}
%    \centering
%    \vspace{1.5em}
%    %\begin{overpic}[width=\columnwidth]{firstImage.jpg}\end{overpic}
%    \begin{overpic}[width=0.45\columnwidth]{firstpicLeft.pdf}\put(28,-10){workspace}\end{overpic}
%    \begin{overpic}[width=0.45\columnwidth]{magneticsetup.pdf}\put(22,-8){magnetic setup}\end{overpic}
%    \vspace{1em}
%    \caption{\label{fig:IntroPic}
%    Workspace and magnetic setup for an experiment to position particles that receive the same control inputs, but cannot move while a control input pushes them into a boundary.
%    } \vspace{-1em}
%    \end{figure}
%\todo{add the picture of magnetic setup}


The paper is arranged as follows. 
After a review of recent related work in Sec.~\ref{sec:RelatedWork},
  Sec.~\ref{sec:angle} introduces angle of repose, a parameter of the particle swarm that can be used for shape control, shown in Fig.~\ref{fig:leadfigure} bottom.
Section \ref{sec:covControl} provides analytical position control results of stable configurations in two canonical workspaces with frictionless walls, shown in Fig.~\ref{fig:leadfigure} top. These results are limited in the set of shapes that can be generated.  To extend the range of possible shapes, the section explores using linear boundary friction.
%describes implementations of the algorithms of covariance control in simulation and hardware experiments. 
 % Sec.  \ref{sec:expResults} 
 Hardware experiments and results are presented in Sec.~\ref{sec:Experiments}.
 We end with directions for future research in Sec.  \ref{sec:conclusion}.



