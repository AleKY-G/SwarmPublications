

\section{Position Control of Two Particles Using Boundary Interaction}\label{sec:PostionControl2Robots}

This section presents algorithms that use non-slip contacts with walls to arbitrarily position two particles in a convex workspace. Our previous work used a square workspace \cite{shahrokhi2017algorithms}. The algorithm can now handle any convex workspace, including the special limit case of a circular workspace. In the last subsection we present techniques to control 3D positioning of two particles.

Workspaces are 2D convex polygons with no internal obstacles. 
 Assume two particles are initialized at $s_1$ and $s_2$ with corresponding goal destinations $g_1$ and $g_2$. 
Denote the current positions of the particles  $p_1$ and $p_2$ 



\subsection{Two Particle Path Planning}


The configuration space for two particles is a four dimensional manifold. Translating both particles the same amount is a trivial operation, but changing the relative positions requires boundary interaction. For this reason, our algorithms use the two dimensional $\Delta$ configuration space, defined as the difference in position of the first particle from the second particle: $\Delta p = p_2 - p_1$.
 Values $.x$ and $.y$ denote the $x$ and $y$ coordinates, i.e., $p_1.x$ and $p_1.y$ denote the $x$ and $y$ locations of $p_1$. 
The algorithm assigns a uniform control input at every instance.
The goal is to move the particles within $\epsilon$ of the goal positions using a shared control input where $\epsilon$ is an arbitrary small number. We do this by first moving them within $\epsilon$ of the correct relative position and then translating the particles to the goal. The relative position is $||\Delta g - \Delta p || = ||(g_2-g_1)- (p_2-p_1)||$.  
 
The $\Delta$ configuration space is a set of all possible $\Delta p$ values.
 $\Delta$ configuration spaces for a representative set of workspaces are shown in Fig.~\ref{fig:polygon}. The \emph{reachable set} is the locus of points in the $\Delta$ configuration space corresponding to any two-move sequence where the first move brings one particle into contact with the boundary, and the second move translates the second particle without moving the first.
 Fig.~\ref{fig:regionMove} shows the starting and ending relative distance as $\Delta s$ and $\Delta g$ in $\Delta$ configuration space% in Fig.~\ref{fig:regionMove}.% The reachable set is the part of $\Delta$ configuration space where if one particle touches a wall in a specific location, the other particle can make the required relative distance without causing the touching particle to move. We will discuss how to make reachable sets in the next subsections. 
 The next subsections give procedures to compute the reachable set.
 Alg.~\ref{alg:optimalAlg} first computes the reachable set. If the goal relative position is in the reachable set, we move particles to achieve relative position. If it is not in the reachable set, we move particles to achieve the closest point on the reachable set from $\Delta g$. 
 \emph{Achieving} a $\Delta$ configuration requires two-moves, the first to push until one particle touches a wall, and the second to adjust the relative spacing.
 Once the correct \emph{relative} position has been achieved, a final translation delivers both particles to their goal destinations. %we go to the final goal position. If it is not our final goal, 
 Otherwise, we iterate until we reach the goal. 
\begin{algorithm}[htb]
\caption{ { \sc 2-ParticlePathPlanning}($s_1,s_2,g_1,g_2,P,\epsilon$)}\label{alg:optimalAlg}
\begin{algorithmic}[1]
%\scriptsize
\Require knowledge of starting $(s_1,s_2)$ and goal $(g_1,g_2)$ positions of  two particles. 
%$(0,0)$ is bottom corner,
 $P$ is a description of the workspace. $\epsilon$ is a positive error bound.
% \emph{PathList} contains all the paths sorted by their path length plus an admissible heuristic. 
% \State  \emph{PathList} $\gets \{\}$
 \State $(p_1,p_2) \gets (s_1,s_2) $ \Comment $p_1 , p_2$ are current positions
\State  moves $\gets \{\}$
% \State $R \gets   \{ p_1,p_2,g_1,g_2  ,\textrm{moves}\} $ \Comment $R$ contains the current particle positions, the goal positions, and the move sequence
 \State $\Delta p \gets p_2-p_1$
 \State $\Delta g \gets g_2-g_1$
\While {$||\Delta p - \Delta g|| > \epsilon$}  %R.p_1 \ne g_1$ \textbf{and} $R.p_2 \ne g_2$}
\State $R_{\textrm{SET}}\gets$  Compute reachable set  \Comment use Alg.~\ref{alg:polygonReachbale} or \ref{alg:circularReachbale}
\State $ \Delta g_c\gets $nearest point in $R_{\textrm{SET}}$ to $\Delta g$
\State $m \gets $move-to-wall corresponding to $\Delta g_c$
\State moves $\gets$ Append $m$ to moves
\State $(p_1, p_2)$ $\gets$ ApplyMove $m$ to $(p_1,p_2)$
%\State $p_2$ = ApplyMove($p_2,m$)
 \State $\Delta p \gets p_2-p_1$
\EndWhile
\State moves $\gets$ Append $g_2-p_2$ to moves \Comment translate to goal
\State \Return moves
\end{algorithmic}
\end{algorithm}


\subsection{Convex Polygonal Workspaces: Reachable Set}
 
 Fig. \ref{fig:polygon} shows different workspaces and their representative $\Delta$ configuration spaces. 


% Consider one particle touching each vertex of the workspace as shown in Fig.~\ref{fig:polygonAlg}.
%  For each pair of vertices, compute where the other robot will be if the first robot goes to that vertex. If the final position of the second robot is still inside the workspace, then its position is one of the vertices of the reachable set. If the point is not inside the polygon, then the intersection of the line that point and the second robot's position with the polygon is one vertex of the reachable set. Compute the distance to all the vertices of the workspace from this point. By subtracting the relative distance of the robots, then all the vertices of one reachable set are found. Doing this for all the vertices of the workspace will give us all the reachable sets.
 \begin{figure}
\centering
\renewcommand{\figwid}{0.8\columnwidth}
{\begin{overpic}[width =\figwid]{differentNumSides.pdf}\put(5,100){Workspace}\put(20,100){$\Delta$ configuration space}
\end{overpic}
}
\caption{\label{fig:polygon}{Reachable sets are drawn with transparent blue polygons. A polygon with $n$ sides has $n$ reachable sets, but if $m$ is even and the polygon is regular, half the reachable sets overlap.
%Workspace and $\Delta$ configuration spaces for different polygonal workspaces and their representative $\Delta$ configuration spaces and reachable sets. As the number of sides in the polygon increases, the total area of the $\Delta$ configuration space is four times of the workspace.
}
\vspace{-1em}
}
\end{figure}


\begin{figure}
\centering
\begin{overpic}[width=0.47\columnwidth]{twoRobotRegionH.pdf}\end{overpic}
\begin{overpic}[width=0.47\columnwidth]{twoRobotRegionV.pdf}\end{overpic}
\caption{\label{fig:TwoRegions}
Boundary interaction is used to change the relative positions of the particles. Each particle gets the same control input. 
(left) If particle 2 hits the bottom wall before particle 1 reaches a wall, particle 2 can reach anywhere along the green line, and  particle 1 can move to anywhere in the shaded area. 
(right) Similarly, if particle 2 hits the right wall before particle 1 reaches a wall, particle 2 can reach anywhere along the green line, and then particle 1 can move to anywhere in the shaded area. 
}
\end{figure}
If a particle is touching a wall, the other particle can move freely in the reachable set as shown in Fig.~\ref{fig:TwoRegions}.
 We compute the reachable set for any convex workspace.
 Fig.~\ref{fig:polygonAlg} illustrates the procedure to construct the reachable set generated by collisions with the $i^{th}$ side. If one particle hist side $i$ before the other ($\overline{ s_1s_2 }\nparallel \overline{ p_i p_{i+1}} $), the reachable set is defined by a polygon, constructed in line 2-13 of Alg.~\ref{alg:polygonReachbale}. The union of these polygons for all $n$ sides is the reachable set of $\Delta$ configurations.
 % by computing the reachable set for all the sides of the polygon. Fig.~\ref{fig:polygonAlg} shows different cases with the particles and an arbitrary side. If particle 2 is touching the $i^{th}$ side, we compute if particle 1 will stay in the polygon or not. Then we generate the reachable set using all the vertices of the polygon and potential place of particle 1. We unify all the reachable sets corresponding to all the sides. Alg.~\ref{alg:polygonReachbale} computes reachable set for any convex polygonal workspace. 
 \begin{algorithm}[htb]
\caption{ { \sc ReachableSetPolygon}($s_1,s_2,g_1,g_2, P$)}\label{alg:polygonReachbale}
\begin{algorithmic}[1]
%\scriptsize
\Require knowledge of starting $(s_1,s_2)$ and goal $(g_1,g_2)$ positions of  two particles. 
$P$ is a list of the vertices of a convex polygon. %RSet contains all the reachable polygons.
\State $R_{\textrm{SET}\gets \{\}}$
\For {$p_i$ in $P$}
\State $p_{i}' \gets s_1 + s_2 - p_i$
\State $p_{i+1}' \gets s_1 + s_2 - p_{i+1}$
\State $L \gets$ line with $(p_{i}', p_{i+1}')$
\State $l_i, l_{i+1} \gets $ intersections of $L$ and polygon $P$
\If {$p_{i}'$ not inside polygon $P$}
\State $p_{i}' \gets l_i$
\EndIf
\If {$p_{i+1}'$ not inside polygon $P$}
\State $p_{i+1}' \gets l_{i+1}$
\EndIf
%\State $D \gets$ All vertices $\in [ l_i , l_{i+1} ] $
\State $D \gets$ $s_2 - s_1 -([l_i, v_{\textrm{min}}, ..., p_i ] -p_i' $,

$[p_{i+1} , o_{i+2}, ... , v_{\textrm{max}}, l_{i+1}] - p_{i+1}')$
%\State $R_{\textrm{SET}} \gets$ Append (polygon ($s_2 - (s_1 +d_i)$),$R_{\textrm{SET}}$ )
\State $R_{\textrm{SET}}$ $\gets$ Append polygon $D$ to $R_{\textrm{SET}}$
\EndFor
\State Return $R_{\textrm{SET}}$
\end{algorithmic}
\end{algorithm}
%\todo{latex line over the top of two vertices}

\begin{figure*}
\centering
\renewcommand{\figwid}{0.66\columnwidth}
\begin{overpic}[width =\figwid]{anypolygon1.pdf}\put(15,0){\tiny{Case 1: $s_1$ does not hit boundary}}\put(28,39){$s_2$}\put(63,50){$s_1$}
\end{overpic}
\begin{overpic}[width =\figwid]{anypolygon2.pdf}\put(25,0){\tiny{Case 2: $s_1$ hits boundary}}\put(20,40){$s_2$}\put(55,58){$s_1$}
\end{overpic}
\begin{overpic}[width =\figwid]{anypolygon3.pdf}\put(25,0){\tiny{Generates polygon defined by $\overline{ p_i p_{i+1}}$ }}
\end{overpic}
\caption{\label{fig:polygonAlg}{ Steps to generate  the reachable set when  one particle collides with edge $i,i+1$ of a convex polygonal workspace.}
\vspace{-1em}
}
\end{figure*}

 
\subsection{Circular Workspaces: Reachable Set}


% Fig.~\ref{fig:shapeControlMathematica1} shows a Mathematica implementation of the algorithm, and is useful as a visual reference for the following description.

To compute the reachable set for a circular workspace, first we consider all possible first contact locations.
 The set of boundary points that a particle can touch before the  other particle  touches is an arc of angle $2(\pi - \frac{\arcsin{d_{12}}}{r})$, where $d_{12}= ||s_1 - s_2||$ and $r$ is the radius of the circle.
 %Here $\alpha$ is the arc of the circle that it is not possible to keep the current relative distance while one robot is touching the wall in that point and is defined by $\alpha = 2 \sin ^{-1}(d)$. 
 We define the angle between two particles as $\theta = \arctan(\frac{p_1.x-p_2.x}{p_1.y - p_2.y})$. 
 
 
\begin{figure}
\centering
\begin{overpic}[width=\columnwidth]{reachableSetCircle.pdf}\end{overpic}
\caption{\label{fig:regionMove}
% the set of points where the red particle is the first to contact the boundary are drawn with a red arc. The  set of points where the blue particle is the first to contact the boundary are drawn with a blue arc. 
Left top: The possible points for the blue and red particles to touch the boundary is shown in blue and red arcs. Left bottom: When the blue particle is touching a wall (blue square) the other particle (red square) can go anywhere in the reachable set (blue region). Right: The $\Delta$ configuration space for the corresponded starting positions of the particles is shown.}
\end{figure}

\begin{figure*}
\centering
\begin{overpic}[width=0.67\columnwidth]{regionMove1.pdf}\end{overpic}
\begin{overpic}[width=0.67\columnwidth]{regionMove2.pdf}\end{overpic}
\begin{overpic}[width=0.67\columnwidth]{regionMove3.pdf}\end{overpic}\\
\vspace{1em}
\centering
\begin{overpic}[width=0.67\columnwidth]{Move1.pdf}\end{overpic}
\begin{overpic}[width=0.67\columnwidth]{Move2.pdf}\end{overpic}
\begin{overpic}[width=0.67\columnwidth]{finalPath.pdf}\end{overpic}
\caption{\label{fig:reachableSet}
Top row shows a polygonal workspace with its reachable sets. Bottom row, left circle shows the workspace. %, and the corresponding reachable set when blue robot is touching the wall is shown as a blue region. 
Right shows the $\Delta$ configuration space and the reachable set that is shown in red is representative of the point we need to go to get to the goal relative distance in one move.%representative of the blue and pink reachable sets shown in the workspace.
}
\end{figure*}

%\begin{figure}
%\centering
%
%\caption{\label{fig:chord}
%When the blue particle is touching a wall (blue square) the other particle (pink square) can go anywhere in the reachable set (blue region).
%}
%\end{figure}
%The algorithm works by expanding the path with the shortest estimated length.
%Expanding a path means either moving directly to the goal, or pushing one particle to a wall and adjusting the relative position of the other particle.
%As soon as the goal is reached, the algorithm returns this path.

A circle has an infinite number of sides, thus infinite reachable sets. However, the reachable set can be parameterized by the angle of first contact location $\psi$, as shown in Fig.~\ref{fig:regionMove} where 
%We make the reachable sets by knowing the angle of the chord, $\psi$, shown in Fig.~\ref{fig:psigamma}. Minimum and maximum values of $\psi$ can be calculated in the following equation.
\begin{align}
 \psi \in [\psi_{\min}, \psi_{\max}]= \Big[\theta +\frac{\sin^{-1}{d}}{2r} - \frac{\pi}{2}, \theta -\frac{\sin^{-1}{d}}{2r} + \frac{\pi}{2}\Big].
% \psi_{\max} &= \theta -\frac{\alpha}{2} + \frac{\pi}{2}
\end{align}
The possible first contact locations are on an arc with interior angle $\gamma$ parameterized by $\psi$ is calculated by:
\begin{align}\label{eq:ppsi}
p_\psi(\psi) &= r[\cos(\psi ), \sin(\psi )]\\ \label{eq:dprep}
d_\perp(\psi)&= 2 ||(s_1.p_\psi(\psi) - s_2.p_\psi(\psi))||\\ \label{eq:gamma}
\gamma(\psi) &= \cos^{-1} \Big(1-\frac{d_\perp(\psi)}{r} \Big) 
\end{align} 
Reachable sets with $\pi$ difference in $\psi$ value are equivalent in the  $\Delta$ configuration space, so we can plan in this space and choose between the two options to immobilize the particle closest to a wall. 
The reachable set for any first contact point defined by $\psi$ is the area under a chord from angle $\psi- \frac{\gamma(\psi)}{2}$ to $\psi+ \frac{\gamma(\psi)}{2}$, for a circle of radius $r$ centered at $c = r(\cos(\psi-\pi), \sin(\psi-\pi))$.

The equations for the four lines outlining the union of reachable sets are as follows:
\begin{align}\label{eq:circlereachable}
l_1 =  r \Big(&(\cos\psi_{\min}- \cos(\gamma + \psi_{\min}) )\\ \nonumber
 + &(\sin\psi_{\min}- \sin(\gamma + \psi_{\min}))\Big) &  0<\gamma< \gamma(\psi_{\min}),\\ \nonumber
l_2 =  r \Big(&(\cos\psi_{\max}- \cos(\gamma + \psi_{\max}))\\ \nonumber
 + &(\sin\psi_{\max}- \sin(\gamma + \psi_{\max}))\Big) &  \gamma(\psi_{\max})<\gamma< 0,\\  \nonumber
l_3 =  r \Big(&(\cos\psi- \cos( \psi+\gamma(\psi) ) )\\ \nonumber
+ &( \sin\psi-\sin( \psi+ \gamma(\psi)))\Big) &  \psi_{\min}<\psi< \psi_{\max},\\ \nonumber
l_4 =  r \Big(&(\cos\psi- \cos( \psi-\gamma(\psi) ))\\ \nonumber
+ & ( \sin\psi- \sin( \psi- \gamma(\psi)))\Big) &  \psi_{\min}<\psi< \psi_{\max}.\\ \nonumber
\end{align}
We combine these boundaries to compute the reachable set. 
Next, find a $\psi$ that would enable us to reach to the required relative goal distance. To do so, we first check $\psi_{min}$ and $\psi_{max}$. Eq.~\eqref{eq:ifinchord} shows if any point $p$, is in the region made by the particles if the touching particle has the angle $\psi$.

\begin{align}\label{eq:ifinchord}
&(p.x - c.x)^2 + (p.y - c.y)^2  > r^2\\ \nonumber
 &(p_x - c_x) \cos\psi + (p.y- c.y) \sin\psi \leq -r\cdot \cos\gamma
\end{align}

%Here $c$ is the coordinate of the workspace center and $r$ is the radius of the workspace.
 If $\Delta g$ is not in the region made by $\psi_{\min}$ or $\psi_{\max}$, we draw a line from $\Delta g$ and the current relative position, $\Delta s$. This line is a chord of the circle and we find the $\psi$ that makes this region. This equation finds this $\psi$:
  \begin{equation}
 \psi = \tan^{-1}(\Delta p - \Delta g ).
 \end{equation}
 
 
\begin{algorithm}[htb]
\caption{ { \sc ReachableSetCircle}($s_1,s_2,g_1,g_2$)}\label{alg:circularReachbale}
\begin{algorithmic}[1]
%\scriptsize
\Require knowledge of starting $(s_1,s_2)$ and goal $(g_1,g_2)$ positions of  two particles. 
%\State $\theta = \arctan(\frac{p_1.x-p_2.x}{p_1.y - p_2.y})$
\State Calculate $p_{\psi}$ \Comment use \eqref{eq:ppsi}
\State Calculate $\gamma$ \Comment use \eqref{eq:gamma}
\State Calculate $l_1, l_2, l_3, l_4$ \Comment use \eqref{eq:circlereachable} 
\State Return $\cup$($l_1, l_2, l_3, l_4$)
\end{algorithmic}
\end{algorithm}
 %Now that $\psi$ is found, we move the particles to make the current goal. If current goal is our final goal, we go to the final goal position. If it is not our final goal, we continue to set the closest point on the reachable set to our current goal until we reach the goal. 
 
% \begin{equation}
% 
% \end{equation}
 

% When a particle contacts a wall, the other particle can move to any position in the reachable set while the first robot is immobile. 
%Two reachable sets are possible, horizontal and vertical. 
%Each non-slip move changes the $\Delta$ configuration space, as shown in  Fig.~\ref{fig:regionMove}.
   

% %This algorithm exploits the position-dependent friction model \eqref{eq:frictionmodel}.
% %employing the assumption we have made earlier about the walls' friction. 
% \subsection{Square workspace}
%Our algorithm uses an A*-like method to find the shortest path in each move. 
%If $(\Delta g.x, \Delta g.y)$ is in the reachable set, one robot touches a wall and the other robot zeros the error in one move. This is shown as $m_2$ in Fig. \ref{fig:reflection}. To find the best place to minimize $m_1$ and $m_3$, the touching robot's goal is reflected on that wall. 
%The minimum distance to get to the goal in two moves when the robot should touch the wall, is the straight line between the robot and the reflection of the goal position on that wall. 
%If the goal configuration can be reached in three moves, then $m_1$  makes one particle hit a wall, $m_2$ adjusts the relative spacing error $\Delta e$ to zero, and  $m_3$ takes the particles to their final positions, as shown in Fig. \ref{fig:shapeControlMathematica1}b. 
%$m_2$ cannot be shortened, so optimization depends on choosing the location where the robot hits the wall. 
% Since the shortest distance between two points is a straight line, reflecting the goal position across the boundary wall and plotting a straight line gives the optimal hit location, as shown in Fig. \ref{fig:reflection}.
%That point is selected when possible, but if this point would cause $m_2$ to push the moving robot out of the workspace, the hit point is translated until the moving robot will not leave the workspace. If $m_2$ causes the two particles to overlap, we add or subtract $\epsilon$ to $m_2.x$ to avoid collisions. This is shown in Fig. ~\ref{fig:epsilon} with three different $\epsilon$ values.
%
%
%\begin{figure}
%\centering
%\begin{overpic}[width=0.47\columnwidth]{twoRobotRegionH.pdf}\end{overpic}
%\begin{overpic}[width=0.47\columnwidth]{twoRobotRegionV.pdf}\end{overpic}
%\caption{\label{fig:TwoRegions}
%Boundary interaction is used to change the relative positions of the robots. Each robot gets the same control input. 
%(left) If robot 2 hits the bottom wall before robot 1 reaches a wall, robot 2 can reach anywhere along the green line, and  robot 1 can move to anywhere in the shaded area. 
%(right) Similarly, if robot 2 hits the right wall before robot 1 reaches a wall, robot 2 can reach anywhere along the green line, and  robot 1 can move to anywhere in the shaded area. 
%}
%\end{figure}
%If  $\Delta g$ is not in the reachable set, we choose the nearest reachable $\Delta x$ and $\Delta y$ to $\Delta g$. 
%%For simplicity in this step, we choose the straight line from the touching robot to the wall for the first move. This may cause the algorithm not to return the shortest path, but the difference is not significant. 
%
%
%Alg.~\ref{alg:optimalAlg} uses an admissible heuristic that adds the current path length to the greatest distance from each robot to their goal. This heuristic directs exploration by expanding favorable routes first.
%\begin{align}\label{eq:admissibleHeuristic}
%h(\text{\emph{moves}}, r_1,r_2,g_1,g_2) =& \sum_{i=1}^{|\text{\emph{moves}}|} \norm{\text{\emph{moves}}_i}  \\
%&+  \max( \norm{g_1-r_1}, \norm{g_2-r_2} ) \nonumber
%\end{align}
%
%
%We  exploit symmetry in the solution by labeling the leftmost (or, if they have the same $x$ coordinate, the topmost) robot $r_1$. 
% If $r_1$ is not also the topmost robot, we mirror the coordinate frame about the right boundary. 
% As an example, consider the two starting positions, $r_1 =  (0.2, 0.2) $ and $r_2 = (0.8, 0.8)$. 
%  Because the leftmost robot is not the topmost robot, we mirror the coordinate frame about the right boundary giving $r_1 = (0.2, 0.8)$ and $r_2 = (0.8,0.2)$. 
% After the path is found, we undo the mirroring to the output path. 
%  Similarly, we exploit rotational symmetry and assume the command pushes a robot to hit the top wall.
%   If a different wall is selected, we rotate the coordinate frame by 90$^{\circ}$, 180$^{\circ}$, or 270$^{\circ}$ counterclockwise and then push the robot to hit the top wall.  After the path is found, we undo the rotation. 
%   This symmetry allows us to use a single function, Alg. \ref{alg:Wallup},  for collisions with all four walls. 
%
%










%\begin{algorithm}[htb]
%\caption{ { \sc 2-ParticleConvexPoly}($r_1,r_2,g_1,g_2,P$)}\label{alg:optimalAlg}
%\begin{algorithmic}[1]
%%\scriptsize
%\Require knowledge of current $(r_1,r_2)$ and goal $(g_1,g_2)$ positions of  two robots. 
%$(0,0)$ is bottom corner,
% $P$ is is a convex polygon. 
% \emph{PathList} contains all the paths sorted by their path length plus an admissible heuristic. 
% \State  \emph{PathList} $\gets \{\}$
% \State $R \gets   \{ h(\{\},r_1,r_2,g_1,g_2  ) ,\{\},r_1,r_2\} $ \Comment $R$ contains $h$, the admissible heuristic \eqref{eq:admissibleHeuristic}, the move sequence, and the current robot positions
%\While {$R.r_1 \ne g_1$ \textbf{and} $R.r_2 \ne g_2$}
%\For{ $\theta \in \{0^\circ, 90^\circ, 180^\circ, 270^\circ \}$ }
%\State $(r_1,r_2,g_1,g_2) \gets$ {\sc Rotate}($R.r_1,R.r_2,g_1,g_2,\theta$)
%\State $\{d, $ \parbox[t]{.3\linewidth}{%
% \emph{moves,}$ r_1,r_2\} \gets$\\
% {\sc  PlanMoveUp}($r_1,r_2,g_1,g_2,L, P.\text{\emph{moves}}$)}
%\State $(\text{\emph{moves}}, r_1,r_2) \gets$ {\sc Rotate}(\emph{moves}$, r_1,r_2,-\theta$)
%\State   {\sc Push} $\{d, \text{\emph{moves}}, r_1,r_2\} $ onto \emph{PathList}
%\EndFor
%\State {\sc Sort}(\emph{PathList}) \Comment sort by admissible heuristic
%\State $R \gets $ {\sc Pop} first element of \emph{PathList}
%\EndWhile
%\State \Return \emph{moves}
%\end{algorithmic}
%\end{algorithm}


%\begin{algorithm}
%\caption{{\sc PlanMoveUp}($r_1,r_2,g_1,g_2,L, $\emph{moves})}\label{alg:Wallup}
%\begin{algorithmic}[1]
%%\scriptsize
%\Require knowledge of current $(r_1,r_2)$ and goal $(g_1,g_2)$ positions of  two robots. 
%$(0,0)$ is bottom corner,
% $L$ is length of the walls. 
% The array \emph{moves} is the current sequence of moves up to the current position.
% Assume $r_1.x < r_2.x$ and $r_1.y \geq r_2.y$. If not, mirror the coordinate frame and swap the robots, then undo the mirroring before returning.
% $\epsilon $ is a small, nonzero, user-specified value.
% 
%\Ensure $(g_1, g_2) , (r_1, r_2)$ all at least $\epsilon$ distance from walls the goals and starting points have at least $\epsilon$ distance from each other.  
%$m_1$ is the first move toward the wall or goal.
% $m_2$ is the second move adjusting $\Delta e$.
%\State $\Delta e \gets (g_2 - g_1)- (r_2- r_1)$
%\If {$\Delta e = (0,0)$} \Comment{base case}
%\State $m_1  \gets g_2 -  r_2$
%\State \emph{moves}$ \gets \{ \text{\emph{moves}},   m_1 \}$
%\State $(r_1,r_2) \gets ${\sc ApplyMove}$(m_1,r_1,r_2)$
%\State \Return $\{h(\text{\emph{moves}}, r_1,r_2,g_1,g_2), \text{\emph{moves}},r_1,r_2\}$
%\EndIf
%\If {$r_2.x - r_1.x - 1 + 2 \epsilon \leq  \Delta g.x \leq  1 ~\textbf{and}~r_2.y - r_1.y \leq \Delta g.y \leq  0$} \Comment{$\Delta g \in $ reachable region}
%\State $m_1 \gets \left(\frac{1-r_1.y}{2-g_1.y-r_1.y} (g_1.x -r_1.x), 1-r_1.y \right)$
%\If {$r_2.x + m_1.x >L$}
%\State $m_1.x \gets 1- r_2.x$
%%\EndIf
%\Else \textbf{ if} {$r_2.x + m_1.x < 0$}  \textbf{then}
%\State $m_1.x \gets -r_2.x$
%\EndIf
%\Else 
%\State $m_1 = (0, 1-r_1.y)$
%\State $\Delta g \gets $ closest reachable $(\Delta x,\Delta y)$.
%%Adjust $\Delta g_x$ and $\Delta g_y$ %\Comment{To go to the nearest $\Delta g_x$ and $\Delta g_y$}
%%\State $\Delta g_x = r_2.x - r_1.x - 1 + 2 \epsilon
%\EndIf
%\State \emph{moves}$ \gets \{ \text{\emph{moves}},   m_1\}$
%\State $(r_1,r_2) \gets ${\sc ApplyMove}$(m_1,r_1,r_2)$
%\State $m_2 \gets \Delta g -  (r_2- r_1)$
%\If {robots on each other \textbf{or} on the wall}
%\State Add $\pm \epsilon$ to $m_2.x$ to avoid collision
%\EndIf
%\State \emph{moves} $ \gets \{ \text{\emph{moves}},   m_2\}$
%\State $(r_1,r_2) \gets ${\sc ApplyMove}$(m_2,r_1,r_2)$
%\State \Return $\{h(\text{\emph{moves}}, r_1,r_2,g_1,g_2), \text{\emph{moves}},r_1,r_2\}$
%\end{algorithmic}
%\end{algorithm}







%
%\begin{figure}
%\centering
%\begin{overpic}[width=0.5\columnwidth]{Reflection.pdf}\end{overpic}
%\caption{\label{fig:reflection}
%If the goal configuration can be reached in three moves, the first move makes one particle hit a wall, the second move adjusts the relative spacing error $\Delta e$ to zero, and the third move takes the particles to their final positions. 
%The second move cannot be shortened, so optimization depends on choosing the location where the robot hits the wall. 
% Since the shortest distance between two points is a straight line, reflecting the goal position across the boundary wall and plotting a straight line gives the optimal hit location.
%} \vspace{-1em}
%\end{figure}

%\begin{figure*}
%\centering
%\renewcommand{\figwid}{0.5\columnwidth}
%{\begin{overpic}[width =\figwid]{epsilonMin.pdf}%\put(20,20){$\epsilon = 0.001$}
%\put(20, 3.1){{\scriptsize$\swarrow$}~$\epsilon = 0.001$}
%\end{overpic}
%\begin{overpic}[width =\figwid]{epsilonMed.pdf}%\put(20,20){$\epsilon = 0.05$}
%\put(20, 3.9){{\tiny$\updownarrow$}~$\epsilon = 0.05$}
%\end{overpic}
%\begin{overpic}[width =\figwid]{epsilonMax.pdf}%\put(20,20){$\epsilon = 0.1$}
%\put(20, 5){{\large$\updownarrow$}~$\epsilon = 0.1$}
%\end{overpic}
%}\\
%\caption{\label{fig:epsilon}{Changing the minimum spacing $\epsilon$ changes the path.   $\epsilon$ is the minimum spacing between two robots and the minimum separation from the boundaries.}
%\vspace{-1em}
%}
%\end{figure*}







\subsection{3D workspaces: Cylinders and Prisms}

Extending path planning to 3D is possible only if the two particles do not initially have the same $x$ and $y$ positions.
For ease of analysis, we assume the workspace boundaries extend in the $\pm z$ direction to form either right cylinders or right prisms.
If the 3D projection is at a different angle, redefine the 2D workspace as a region perpendicular to the projection.
 First, we move the closest particle to the boundary, which prevents its $z$ coordinate from changing.  
 We next apply actuation in either the $\pm z$ direction to achieve the desired $\Delta z$.
 Then the particles are actuated away from the boundary and to the appropriate $z$ positions.
 Path planning continues using Alg.~\ref{alg:optimalAlg} to position the particles to the desired $x$ and $y$ positions. 
 As an example, consider Fig.~\ref{fig:zaxis} which shows a cylindrical workspace.
 The blue particle starts in the blue disk and the red particle starts in the red disk. 
 The two candidate shortest-length paths that touch the wall are shown with parallel arrows. 
Each arrow will cause one of the particles to touch the wall, enabling the other particle to move freely in the  $z$ axis to achieve the required relative position.
This can be extended to other 3D workspaces if the workspace can be locally approximated as a 3D prism or cylinder. Other workspaces may be better handled by other path planners, such as \cite{AaronManipulation2013}, which used  collisions with  protrusions of the workspace to rearrange particles.

\begin{figure}
\centering
\begin{overpic}[width=0.5\columnwidth]{zaxis.pdf}\end{overpic}
\caption{\label{fig:zaxis}
Extending the algorithm to position the particles in 3D.
} %\vspace{-1em}
\end{figure}






