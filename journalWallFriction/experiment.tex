%%%%%%%%%%%%%%%%%%%%%%%%%%%%%%%%%%%%%%%%%%%%%%%%%%%%%%%%%%%
\section{Experimental Results}\label{sec:expResults}
%%%%%%%%%%%%%%%%%%%%%%%%%%%%%%%%%%%%%%%%%%%%%%%%%%%%%%%%%%%
\begin{figure*}[htb!]\label{fig:3dPrinted}
\centering
\vspace{1.5em}
%\begin{overpic}[width=\columnwidth]{firstImage.jpg}\end{overpic}
\begin{overpic}[width=2\columnwidth]{3dexperiment.pdf}\end{overpic}
\\
\vspace{1em}
\begin{overpic}[width=2\columnwidth]{realTripe.pdf}\end{overpic}
\caption{\label{fig:story}
Frames showing particle positions before and after control inputs. Top row: small intestine phantom. Bottom row: cow stomach tissue.
} \vspace{-1em}
\end{figure*}

To demonstrate Alg.~\ref{alg:optimalAlg} experimentally, we performed several tests.
Each used the same magnetic setup shown in Fig.~\ref{fig:IntroPic}.
% The figure might need to be referenced 
 Two different intestine models were employed, the first a 3D-printed cross-section representation of a small intestine, and the second a cross-section of a bovine stomach.
 
 \subsection{Magnetic Manipulation Setup}
 
 The magnetic manipulation system has two pairs of electromagnetic coils, each with iron cores at their centers, and arranged orthogonal to each other. The iron core at the center of each coil concentrates the magnetic field towards the workspace. An Arduino and four SyRen regenerative motor drivers were used for control inputs to the coils. Finally, a FOculus F0134SB 659 x 494 pixel camera was attached to the top of the system, focusing on the workspace which was backlit by a $\SI{15}{\watt}$ LED light strip. 
 
To obtain experimental data, the test samples (the phantom intestine model and the bovine cross section) were placed in laser-cut acrylic discs and then immersed in corn syrup. Corn syrup was used to increase the viscosity to 12000 cP for the experiments. Spherical $\SI{1}{\milli\metre}$ magnets (supermagnetman \#SP0100-50) were used as our particles. Our experimental setup did not perfectly implement the system dynamics in \eqref{eq:swarmDynamicsAndFric}. In particular, the magnetic field in this setup is only approximately uniform. The magnetic force varies in both magnitude and orientation. As shown in the video attachment, this non-uniformity causes the particle closer to the coil to move faster than the other particle. This phenomenon makes it easier to increase particle separation than to decrease separation, but boundary collisions can be used to decrease separation. Also, magnetic forces are not exactly parallel, but point toward the center of the activated coil. Algorithm~\ref{alg:optimalAlg} is robust to these non-uniformities, but sometimes requires additional iterations.
 
%explain the magnetic field strength used, the camera system (briefly)
%explain the particles used

%explain the fluid used in the model.

\subsection{Intestine Phantom Model}

The intestine phantom model was used first and was made to mimic the geometry of an intestine and its villi. The model consists of a circular ring with an outer diameter of $\SI{50}{\milli\metre}$, an inner diameter of $\SI{46}{\milli\metre}$, and 60 $\SI{2}{\milli\metre}$ long protrusions on its inner surface cut out of $\SI{6}{\milli\metre}$ thick acrylic to model the geometry of intestinal villi. Figure \ref{fig:story} top row shows an experiment. Starting and ending positions were printed beneath the workspace on transparency film. Our algorithm successfully delivered the particles to goal positions in ?? out of ?? trials.

%methodology: we used xx material, built a model xx big


%todo: snapshots of the the beads in the model, moving toward goal.  We can add lots of annotations.  %Fake this series so we have an image for now.


\subsection{Bovine Stomach Cross-section}
%todo: procedure for fixating the Bovine tissue sample,
%discussion of the challenges 

Strips of cow stomach approximately $\SI{5}{\milli\metre}$ thick were cut and sewn to acrylic cylinder and then glued to an acrylic substrate using cyanoacrylate (superglue). This assembly was then filled with corn syrup. The experiment is shown in Fig.~\ref{fig:story} bottom row. Our algorithm successfully delivered the particles to goal positions in ?? out of ?? trials.
A video showing one trial of this experiment is available in the supplementary materials. 
%In using the small intestines, which were about 25mm in diameter, the resulting workspace was less than half of that of our simulated intestines. This meant more care had to be taken in navigating the magnets to their goal locations to avoid getting them too close to each other. 

%todo: snapshots of the the beads in the model, moving toward goal.  We can add lots of annotations.  Fake this series so we have an image for now.


