
%%%%%%%%%%%%%%%%%%%%%%%%%%%%%%%%%%%%%%%%%%%%%%%%%%%%%%%%%%%
\section{Experimental Results}\label{sec:expResults}
%%%%%%%%%%%%%%%%%%%%%%%%%%%%%%%%%%%%%%%%%%%%%%%%%%%%%%%%%%%

To demonstrate Alg. XX experimentally, we performed several tests.
Each used the same magnetic setup.
 Two different intestine models were employed, the first a 3D-printed cross-section representation of a small intestine, and the second bovine small intestine.
 
 \subsection{Magnetic Manipulation Setup}
 explain the magnetic field strength used, the camera system (briefly)

todo: image of the magnetic setup with scale bar
explain the particles used

explain the fluid used in the model.

\subsection{Intestine Phantom Model}

methodology: we used xx material, built a model xx big


todo: snapshots of the the beads in the model, moving toward goal.  We can add lots of annotations.  Fake this series so we have an image for now.


\subsection{Bovine Intestine Cross-section}
todo: procedure for fixating the Bovine tissue sample,
discussion of the challenges 

todo: snapshots of the the beads in the model, moving toward goal.  We can add lots of annotations.  Fake this series so we have an image for now.


\begin{figure*}
\centering
\vspace{1.5em}
%\begin{overpic}[width=\columnwidth]{firstImage.jpg}\end{overpic}
\begin{overpic}[width=0.4\columnwidth]{story1.png}\end{overpic}
\begin{overpic}[width=0.4\columnwidth]{story2.png}\end{overpic}
\begin{overpic}[width=0.4\columnwidth]{story3.png}\end{overpic}
\begin{overpic}[width=0.4\columnwidth]{story4.png}\end{overpic}
\caption{\label{fig:story}
Positioning particles that receive the same control inputs, but cannot move while a control input pushes them into a boundary.
} \vspace{-1em}
\end{figure*}

