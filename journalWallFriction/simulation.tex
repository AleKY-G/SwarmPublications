
\section{Simulation}\label{sec:simulation}


%Two simulations were implemented using non-slip contact walls for position control.  The first controls the position of two robots, the second controls the position of $n$ robots.  

%\subsection{Position Control of Two Robots}

Algorithm \ref{alg:optimalAlg}  was implemented in Mathematica using particles with zero radius. 
%An online interactive demonstration and source code of the algorithm are available at \cite{Shahrokhi2015mathematicaParticle}.
%  Fig.~\ref{fig:shapeControlMathematica1}  shows  an implementation of this algorithm with robot initial positions represented by hollow squares and final positions by circles. 
 %Dashed lines show the shortest route if robots could be controlled independently, while solid lines show the optimal shortest  path using uniform inputs.
 
 The contour plots in Fig.~\ref{fig:contour} left shows the length of the path for given $s_1,s_2,g_1$ with $g_2$ ranging over all the workspace. Fig.~\ref{fig:contour} right shows the total distance of the path. This plot clearly shows the nonlinear nature of the path planning. The hardest point to achieve is the when the goals have $\pi$ difference and are very close to the boundary. Fig.~\ref{fig:contourPlots} shows the same concepts in a square workspace. Fig.~\ref{fig:contourPlots} top row considers the particles are interchangeable for three arbitrary starting and goal positions for the particles. Using interchangeable particles makes the path length significantly smaller. The middle and bottom row shows the same configurations with unique particles.
 %If the length of each side of the square workspace is $L$, the worst case path length is $(\sqrt{2}+2)L$.
 
 The plots in Fig.~\ref{fig:deltanumdist} show the exponentially increasing number of moves and distance when the accuracy of reaching to the goal ($\delta$) is getting to zero when the goal positions have $\pi$ difference with each on the boundaries.



\begin{figure}
\centering
\begin{overpic}[width=0.49\columnwidth]{numcontour.pdf}\end{overpic}
\begin{overpic}[width=0.49\columnwidth]{distcontour.pdf}\end{overpic}
 \vspace{-2em}
\caption{\label{fig:contour}
Plots show performance with one goal on the boundary.
}
\end{figure}

\begin{figure}
\centering
\begin{overpic}[width=0.49\columnwidth]{deltanum.pdf}\end{overpic}
\begin{overpic}[width=0.49\columnwidth]{deltadist.pdf}\end{overpic}
\caption{\label{fig:deltanumdist}
Particle swap poles as their goal is the worst case. It can never be achieved but can be asymptotically reached. Plots show decreasing error as the number of moves grows.
} 
\vspace{-1em}
\end{figure}


\begin{figure}
\centering
\renewcommand{\figwid}{1\columnwidth}
{
\begin{overpic}[width =\figwid]{contourDistnew.png}\put(-2,10){\begin{turn}{90} \tiny{unique particles}
\end{turn}}

\end{overpic}
\vspace{1em}
\begin{overpic}[width =\figwid]{JustSimulationV6.png}\put(-2,10){\begin{turn}{90} \tiny{unique particles}
\end{turn}}

\end{overpic}
\begin{overpic}[width =\figwid]{identical.png}\put(-2,6){\begin{turn}{90} \tiny{interchangeable particles}
\end{turn}}
\end{overpic}
}\caption{\label{fig:contourPlots}{Starting positions of particles $1$ and $2$ and goal position of particle $2$ are fixed, and $\epsilon=0.001$.
 The top row of contour plots show the distance if robot $1$'s goal position is varied in $x$ and $y$. The bottom row shows the number of moves required for the same configurations.}
\vspace{-1em}
}
\end{figure}






