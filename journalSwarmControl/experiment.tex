
%%%%%%%%%%%%%%%%%%%%%%%%%%%%%%%%%%%%%%%%%%%%%%%%%%%%%%%%%%%
\section{Object manipulation results}\label{sec:exp}
%%%%%%%%%%%%%%%%%%%%%%%%%%%%%%%%%%%%%%%%%%%%%%%%%%%%%%%%%%%

This section analyzes a \emph{object manipulation} task attempted by both our hybrid, hysteresis-based controller and by human users.  

\subsection{Human-controlled object manipulation}



As we saw in previous section, players with just the mean completed the task faster than those with full-state feedback.  As Fig.~\ref{fig:ResVaryControlVis}.b shows, the levels of feedback arranged by increasing completion time are [mean, mean + variance, full-state, convex-hull].  Interviews with  beta-testers suggests that tracking 100 robots was overwhelming---similar to schooling phenomenons that confuse predators---while working with just the mean + variance was like using a ``spongy'' manipulator. Convex-hull feedback was confusing and irritating because a single robot left behind an obstacle would distort the entire hull, obscuring the information about the majority of the swarm.
%obscuring what the rest of the swarm is doing.   


\subsection{Automated object manipulation (simulation)}
Fig.~\ref{fig:story} shows snapshots during an execution of this algorithm. To solve this object manipulation task, we discretized the environment. On this discretized grid we used breadth-first search to determine $\mathbf{M}$, the shortest distance from any grid cell to the goal, and generated a gradient map $\nabla \mathbf{M}$ toward the goal as shown in Fig.~\ref{fig:BFSGradient}.  The object's center of mass is at $\mathbf{b}$ and has radius $r_b$. 
Three constants are needed, where $k_1>k_2>1$ and $1>k_2>0$. All experiments used $[k_1,k_2,k_3] = [2.5,1.5,0.1]$.
The robots were directed to assemble behind the object at  $\mathbf{b} - k_2 r_b \nabla \mathbf{M}(\mathbf{b})$, then move to  $\mathbf{b} - k_3 r_b \nabla \mathbf{M}(\mathbf{b})$ to push the object toward the goal location. We use the hybrid hysteresis-based controller in Alg.~\ref{alg:MeanVarianceControl}  to track the desired position, while maintaining sufficient robot density to move the object by switching to minimize variance whenever variance exceeds a set limit. The minimize variance control law \eqref{eq:PDcontrolVariance} is slightly modified to choose the nearest corner further from the goal than $\mathbf{b}$ with an obstacle-free straight-line path to $\mathbf{b}$. 
The control algorithm  for object manipulation is listed in Alg.~\ref{alg:BlockPushing}. 
Experimental results are summarized in Fig.~\ref{fig:AutoVeryParam}.a. It is shown that large number of robots will complete the task faster, although after some point adding more robots would not cause a significant improvement. 

\begin{algorithm}
\caption{Object-manipulation controller for a robotic swarm.}\label{alg:BlockPushing}
\begin{algorithmic}[1]
\Require Knowledge of swarm mean $[\bar{x},\bar{y}]$, variance $[\sigma_x^2, \sigma_y^2]$,  moveable object's center of mass $\mathbf{b}$, map of the environment, and the locations of all convex corners $\mathbf{C}$
\Require Robot distribution is unimodal
\Require Obstacle-free, straight-line path from swarm to moveable object
\State Compute $\mathbf{M}$, the distance to goal, with breadth-first search
\State Compute the gradient, $\nabla \mathbf{M}$
\State $\mathbf{C} \gets \mathrm{sort(\mathbf{C})}$ according to $-\mathbf{M}$
\While{$\mathbf{b}$ is not in goal region}
\State $\sigma^2 \gets \max{(\sigma_x,\sigma_y)}$
\If {$\sigma^2 > \sigma_{max}^2$}
\While{$\sigma^2 > \sigma_{min}^2$}
\State $\mathbf{c}_i \gets$ the nearest corner in $\mathbf{C}$ to $[\bar{x},\bar{y}]$
\State $ [x_{goal}, y_{goal}] \gets \mathbf{c}_i $
\If {$\mathbf{M}(\mathbf{b}) > \mathbf{M} (\mathbf{c}_i)$}
\State  $[x_{goal}, y_{goal}] \gets  \mathbf{c}_{i-1}$ 
\State Apply \eqref{eq:PDcontrolPosition} to move toward $[x_{goal}, y_{goal}]$
\EndIf
\EndWhile
\Else  
\If {$\mathrm{distance}( \mathbf{b}, [x_{goal}, y_{goal}] ) > k_1 r_b$}
	\State$r_p \gets k_2 r_b$  \Comment{guarded move}
	\Else
	\State$r_p \gets k_3 r_b$  \Comment{pushing move}
	\EndIf
\State $[x_{goal}, y_{goal}] \gets \mathbf{b} - r_p \nabla \mathbf{M}(\mathbf{b})$ 
\EndIf
\State Apply \eqref{eq:PDcontrolPosition} to move toward $[x_{goal}, y_{goal}]$
\EndWhile
\end{algorithmic}
\end{algorithm}



\begin{figure*}
\centering
%\renewcommand{\figwid}{0.19\columnwidth}
%\href{http://youtu.be/tCej-9e6-4o}{\begin{overpic}[width =\figwid]{story1.png}\put(6,15){T = 5 s}
%\end{overpic}
%\begin{overpic}[width =\figwid]{story2.png}\put(6,15){T = 12 s}
%\end{overpic}
%\begin{overpic}[width =\figwid]{story3.png}\put(6,15){T = 20 s}
%\end{overpic}
%\begin{overpic}[width =\figwid]{story4.png}\put(6,15){T = 25 s}
%\end{overpic}
%\begin{overpic}[width =\figwid]{story5.png}\put(6,15){T = 33 s}
%\end{overpic}}
\begin{overpic}[width =\columnwidth]{SwarmRun.pdf}
\end{overpic}
\vspace{-1em}
\caption{\label{fig:story}\href{http://youtu.be/tCej-9e6-4o}{Snapshots showing an object manipulation simulation with 100 robots under automatic control.  See the video attachment for an animation~\cite{ShivaVideo2015}.}
%\vspace{-2em}
}
\end{figure*}

\begin{figure}
\centering
\begin{overpic}[scale=0.19]{BFSMode.png}
\end{overpic}
\begin{overpic}[scale=0.19]{GradientView.png}
\end{overpic}
\begin{overpic}[scale=0.19]{PolicyIter.png}
\end{overpic}
\vspace{-1em}
\caption{\label{fig:BFSGradient}The BFS algorithm finds the shortest path for the moveable object (left), which is used to compute gradient vectors (middle). Using policy iterations enables encoding penalties for being near obstacles (right).
%\vspace{-2em}
}
\end{figure}

\begin{figure}
\centering
%\begin{overpic}[width=\columnwidth]{Snapshots.pdf}\put(6,29){\emph{t} = 3 s}\put(38,29){\emph{t} = 410 s}\put(70,29){\emph{t} = 710 s}
%\put(15,22){\emph{t} = 1374 s}\put(49,22){\emph{t} = 2185 s}\put(83,22){\emph{t} = 2703 s}
%\end{overpic}
\begin{overpic}[width=.9\columnwidth]{Snapshots.pdf}\put(6,80){\emph{t} = 3 s}\put(38,80){\emph{t} = 410 s}\put(70,80){\emph{t} = 710 s}
\put(15,8){\emph{t} = 1374 s}\put(49,8){\emph{t} = 2185 s}\put(83,8){\emph{t} = 2703 s}
\end{overpic}

\vspace{-1em}
\caption{\label{fig:expSnapShot}{Snapshots showing the object manipulation experiment with 100 kilobots under automatic control. The automatic controller will see the pink objects as obstacles and finds its path to the goal. See the video attachment for an animation~\cite{ShivaVideo2015}}
%\vspace{-2em}
}
\end{figure}



\begin{figure*}
\centering
\renewcommand{\figwid}{0.5\columnwidth}
\begin{overpic}[width =0.45\columnwidth]{SimVeryNum.pdf}\put(1,60){a)}
\end{overpic}
\begin{overpic}[width =0.45\columnwidth]{SimVeryNoise.pdf}\put(1,60){b)}
\end{overpic}
\begin{overpic}[width =0.45\columnwidth]{SimVeryWeight.pdf}\put(1,65){c)}
\end{overpic}
\begin{overpic}[width =0.45\columnwidth]{SimVeryShape.pdf}\put(1,65){d)}

\end{overpic}
\vspace{-1em}
\caption{\label{fig:AutoVeryParam}Parameter sweep for a) number of robots, b) different noise values, c) object weight, and d) object shape.  Each bar is labelled with the number of trials.
%\vspace{-2em}
}
\end{figure*}






%Algorithm \ref{alg:BlockPushing} is an imperfect solution and has a failure mode if the robot swarm becomes multi-modal with modes separated by an obstacle, as shown in Fig.~\ref{fig:Failure}.  In this case, moving toward a corner will never reduce the variance below $\sigma_{min}^2$.


  The first challenge is to identify when the distribution has become multi-modal.  Measuring just the mean and variance is insufficient to determine if a distribution is no longer unimodal, but if the swarm is being directed to a corner, and the variance does not reduce below $\sigma_{min}^2$, the swarm has become separated. In this case, we must either manipulate with a partial swarm, or run a gathering algorithm.  For the  {\sffamily S}-shaped workspace in this study, an open-loop input that commands the swarm to move in succession \{{\sc west, north, east, south}\} will move the swarm to the bottom right corner.
This is not true for all obstacle fields. In a {\sffamily T}-shaped workspace, it is not possible to find an open-loop input that will move the entire swarm to the bottom of the {\sffamily T}.  
 
  Using only the mean and variance may be overly restrictive.  Many heuristics using high-order moments have been developed to test if a distribution is multimodal~\cite{haldane1951simple}.  Often the sensor data itself, though it may not resolve individual robots, will indicate multi-modality.  For instance CCD images reveal clusters of bacteria, and MRI scans show agglomerations of particles~\cite{stuber2007positive}.  This data can be fitted with $k$-means or expectation maximization algorithms, and manipulation could be performed with the nearest swarm of sufficient size.
  








