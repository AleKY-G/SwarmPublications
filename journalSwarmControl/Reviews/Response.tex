\documentclass[conference, onecolumn, 10pt]{../IEEEtran}
\IEEEoverridecommandlockouts
\usepackage{times}


\usepackage{microtype,xparse,tcolorbox}
\usepackage{mathtools, cuted}
\usepackage{todonotes}

\newcounter{reviewer}
\setcounter{reviewer}{0}
\newcounter{point}[reviewer]
\setcounter{point}{0}

\setlength\parindent{0pt}

\renewcommand{\thepoint}{[R\,\thereviewer.\arabic{point}] } 

\newcommand{\response}[1]{\refstepcounter{point}  \smallbreak \noindent
	\hangindent=3.5em
	\hangafter=1
	\thepoint \space ``\textcolor{blue}{#1}" \par\smallskip}

\NewDocumentCommand \Reviewer { m } {
	\stepcounter{reviewer}
	\subsection*{Comments~by~Reviewer~#1}
}

\begin{document}
	
	\vspace*{\fill} 
	\begin{quote}  
		\todo[inline]{I still need to format this page to look like the one in the word document}
		
		Steering a Particle Swarm Using Global Inputs and Swarm Statistics
		
		
		Dear IEEE Transactions on Robotics Editorial Office,
		
		Please find attached the revised paper, Steering a Particle Swarm Using Global Inputs and Swarm Statistics
		along with the document containing response to the reviewers. We are grateful to the reviewers for helping us in improving our manuscript through their comments and questions. Please let us know if further information is required.
		
		
		Sincerely,
		
		\todo[inline]{Insert Signature}
		
		
		Aaron T Becker (on behalf of all the authors) 
	\end{quote}
	\vspace*{\fill}
	\break 
	
	\section*{Response to Reviewers}
	In the following document, we’ve provided detailed responses to the comments and questions of the reviewers. Comments and questions by reviewers are in \textcolor{blue}{blue}, our responses are in black.
	
	\Reviewer{\#1}\response{the paper is very interesting, well written and well organized. It presents an important research topic which is of great interest of most people interested in swarm robotics. The video is also very nice and useful to better understand the work.}
	Thank you!
	
	\response{What are the assumptions on the manipulated block properties? (Friction? Necessary force to move it? Is one robot enough to move it?)}
		\todo[inline]{This is a question we were trying to avoid right? Since we can't seem to measure the strength of a single kilobot. Should we give simulation specs? What about the real world experiments?}
	  
	\response{Regading the hybrid controller, what are the necessary assumptions to reach convergence? Are there practical assumptions to guarantee the claim (18) =$>$ (19)? How about the geometry of the environment? What happens if the environment is sparse so that corners are too far? Is it possible to control variance in this case?}
	This is an interesting problem that we can expand on in a later work.
		
	\response{In section VI the authors should give more detils regarding the Value Iteration approach. How did the authors set the probabilities p($x_j$ | x,u)? It seems it should be j instead of i in the loop "for i = ..." in (25).}
	The probabilities for set p($x_j$ | x,u) come from the reward function in equation (23). The loop in equation (25) is looping the probability p($x_i$ | $x_j$,u) such that for a specific $x_j$ it iterates through all $x_i$.
		
	\response{It is also necessary to give more details of the potential fields in VI-b. The parameters $\Delta \rho$, $\rho$, b in (26) and (27) were not defined.}
	A sentence has been added and a second sentence has been altered for clarification, now reading, "... The repulsive potential field is centered at the object's COM and is active when the swarm mean is within $\theta$ of the desired direction of motion $\mathbf{D}(\mathbf{b})$ and $\rho$, the distance between the swarm mean and the object's COM, is less than $\rho_0$ as shown in Fig.~\ref{fig:potentialField}b. ..."
	
	\response{In the outlier rejection, how are the regions computed?}
		\todo[inline]{We already state, "The \emph{main} regions are generated by extending obstacles until they meet another obstacle. The \emph{transfer} regions are perpendicular to obstacle boundaries, and act as a buffer between two main regions." Are they looking for a formula?}
		
	\response{In algorithm 2, how is (29) applied? Is it used directly as ux and uy? The composition of the potential field and the plan obtained with the value iteration algorithm should be better clarified.}
	Yes, We have amended the paragraph prior with "These fields operate in place of $\mathbf{D}(\mathbf{b})$ only when the repulsive field is active."
		
	\response{In the simulations, what is the model for collisions? }
	We use a Javascript port called Box2D to calculate our physics. This has been clarified in Section 5.

	\Reviewer{\#2}Is there not a Reviewer 2?

	\Reviewer{\#3}\response{In the paper the terms: micro, nano robots, and particles are used interchangeably, that is perfectly fine, however, it would be better in the very beginning to make that clear possible with a footnote, especially the use of the term particles. }
		\todo[inline]{I thought in the first paragraph we state that our robots are the particles?}
		
	\response{In page two, column 1 , the authors state that "process noise cancels the differentiating effects... tens of robots." The term tens of robots is not clear. It would also help a citation if it exists.}
		\todo[inline]{Doesn't the term "tens of..." mean something of magnitude X$*10$?}
		
	\response{Throughout the paper citations [?] are used as nouns, while that is quite common practice the paper reads better when the authors are used. e.g. "[123] shows that..." instead use: "Joe and Bob [123] show that..."  }
		\todo[inline]{Should I change all of our citations then?}
		
	\response{When the walls are immovable then the proposed technique to push the swarm against a corner works as is demonstrated. In the case of the human body where the walls could be more elastic, or even permeable, how that would affect the proposed strategy.}
	That is a good question which we would love to observe further in a future work.
		
	\response{In the same line, if pushing against a "wall" could increase friction or even absorption of the transported medicine, have you consider a strategy that envelops the object with a small group of particles shielding the object from the walls?}
	That is a good idea however we are unable to implement such an idea in the current iteration of this paper.
		
	\response{The reviewer will recommend in simulation to create bigger more complex environments emulating the major veins or the digestive tract of the human body. Definitely something with curvy walls. }
	That is a good idea however we are unable to implement such an idea in the current iteration of this paper.
		
	\response{The paper, and in general the proposed approach, will gain by demonstrating the proposed approach in environments where there are junctions where the object has to be pushed in one of two, or more, options. }
	We are currently working on attempting to do just this! Look out for our future work.
	
	\response{Furthermore, it would be interesting to see if the proposed method will work in an open environment with multiple obstacles.}
	We would love to expand on our work in this fashion and are looking for collaborators to do so.

	\Reviewer{\#4}
	\response{One major concern is about the Lyapunov derivative in eq. (17) which looks wrong. I found $\dot{V}=2 \sigma \dot{\sigma} (\sigma^2-\sigma_{goal}^2)$. This demands to revise your justification as your result on variance stability conditions might be totally different. Please clarify.}
	\todo[inline]{I will look into this one more later, maybe.}
		
	\response{The global control law is presented for holonomic robots, while in small-scales often particles obey nonholonomic motion. Is your approach still applicable? }
		\todo[inline]{I don't know and if I did, is s/he asking for us to add in the paper?}
		
	\response{In eq. (17) how can be $\dot{\sigma}^2$ be negative? Please clarify.}
		\todo[inline]{That.... is a good question.}
		
	\response{Eq. (20) is a little bit vague. A clearer statement is required. Perhaps to say, eq. (20) is an extended version of eq. (11), which allows to control variance as well.}
	Great idea! The paper now reads, "A PID controller based on the original PD controller in Eq. (11) to regulate the variance to $\sigma^2_{\rm{ref}}$ is:".
		
	\response{Page 3 section 3 on the lesson list number 2, you have to define IID?}
	This issue has been remedied entirely by changing IID to "independent and identically distributed".
		
	\response{Page 3 in section 3A, please follow paragraph style in the middle of the text.}
		\todo[inline]{Dr Becker? I don't understand what isn't in paragraph format?}
		
	\response{In section 3C and D, the writing can be improved to make it easier for general readership.}
		\todo[inline]{It doesn't give much in terms of what it lacks though.}
		
	\response{Correct the repetition (control control) on the last line on the left in page 5.}
	Corrected, the line now reads, "There are several techniques for breaking the symmetry of the control input to allow controlling more states, ..."
		
\end{document}