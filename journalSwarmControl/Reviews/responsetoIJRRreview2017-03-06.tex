1. Change the names of papers in the beginning of the paper.  This helps editor select better reviewers.
2. check that we cite the papers 




06-Mar-2017

Dear Ms. Shahrokhi:

I write you in regards to manuscript # IJR-16-2710 entitled "Steering a Swarm Using Global Inputs and Swarm Statistics" which you submitted to the International Journal of Robotics Research.

In view of the criticisms of the reviewer(s) found at the bottom of this letter, your manuscript has been denied publication in the International Journal of Robotics Research.

Thank you for considering the International Journal of Robotics Research for the publication of your research.  I hope the outcome of this specific submission will not discourage you from the submission of future manuscripts.

Sincerely,
Prof. John Hollerbach
Editor, International Journal of Robotics Research
Ijrr.admin@sagepub.co.uk


Reviewer(s)' Comments to Author:

Reviewer: 1

Comments to the Author
The control of robot swarms using global input, and the underlying requirements in terms of human-swarm interactions, is an important area of research. The results presented here are exciting and impressive.

However, the theory and all figures up until Figure 12 were taken directly from, or are closely based on the following papers by the authors:

�Crowdsourcing Swarm Manipulation Experiments: A Massive Online User Study with Large Swarms of Simple Robots�

�Stochastic Swarm Control with Global Inputs�

This leaves only 3 pages (10-12) with original unpublished content and therefore does not warrant a full journal publication. The paper should be rewritten to focus on the original material.

Reviewer: 2

Comments to the Author
This paper describes controllers for steering a swarm using global inputs for object transportation. The paper is well written and its contribution is important. In particular, experimentally demonstrating that control of a large swarm is indeed possible using simple global control input has a number of applications in micro and nano robotics for drug delivery.

Unfortunately, there are two major problems with the paper, which have to be addressed prior to publication. First, the control of a robot swarm using simple global inputs and the resulting distributions from biased random walk are actually already quite well understood and described in the references below. The novelty of this paper lies therefore in leveraging these techniques for object transport. I recommend the authors to ground their theoretical models and controllers - which might have indeed advantages - as much as possible in the above literature. Second, although I really, really like the human swarm interaction experiments, their connection to the robot controller and using them as a general method for design inspiration is questionable from a scientific perspective. It is akin to let humans estimate the square root of a number and then extract heuristics for doing so using a computer. What might be hard for people could be very easy for a computer and the other way round. Reducing the dimensionality of the problem is actually the first thing a control designer would do, possibly in a way that can be mathematically justified. Such an analysis might then reveal that using the median would be even better than using the mean as it is more robust to outliers. The approach is also at odds with the principled method provided by Milutinovic (see references below) and should probably not be in the same paper as any formal method as long as there cannot be made a clear connection between the two. My suggestion to resolve this problem would be to motivate your choice of information reduction options using the existing literature, and then focus exclusively on the object transportation task. Another option could be to use the simulation results to tune parameters of the controller.


REFERENCES

Milutinovic, Dejan, and Pedro Lima. "Modeling and optimal centralized control of a large-size robotic population." IEEE Transactions on Robotics 22.6 (2006): 1280-1285.

Bretl, Timothy. "Control of Many Agents Using Few Instructions." Robotics: Science and Systems. 2007.

Palmer, Aaron, and Dejan Milutinovi?. "A hamiltonian approach using partial differential equations for open-loop stochastic optimal control." American Control Conference (ACC), 2011. IEEE, 2011.

Prorok, Amanda, Nikolaus Correll, and Alcherio Martinoli. "Multi-level spatial modeling for stochastic distributed robotic systems." International Journal of Robotics Research 30.5 (2011): 574-589.

Demir, Nazl�, and Beh�et A��kme?e. "Probabilistic density control for swarm of decentralized ON-OFF agents with safety constraints." American Control Conference (ACC), 2015. IEEE, 2015.

Associate Editor Comments:

Associate Editor
Comments to the Author:
There is agreement among both reviewers that the work presented in this manuscript is of high quality and of interest to the readership of IJRR. However, the reviewers also agree in their assessment of the manuscript's novelty with respect to prior work by the authors and by others in the community. Reviewer 1 indicates that the novelty beyond the author's prior work is insufficient to warrant a journal publication. Reviewer 2 is enthusiastic about the work but suggests that the novelty must be presented in context of related work by others to be assessed properly. Some of the references provided by reviewer 2 closely relate to theoretical contributions in the manuscript. It would therefore be necessary to clearly explain how the proposed methods extend the state of the art, as represented in those additional references. As the correction of both criticisms will require substantial changes to the manuscript and will probably necessitate the inclusion of additional theoretical and experimental results, the IJRR review process requires that the manuscript be rejected at this time. However, after you have carefully addressed the reviewers' concerns, you are encouraged to submit a new manuscript to IJRR. This new manuscript will be handled independently of this submission.