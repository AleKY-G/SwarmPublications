

\section{Position Control of Two Robots Using Boundary Interaction}\label{sec:PostionControl2Robots}


Alg.~\ref{alg:optimalAlg} uses non-slip contacts with walls to arbitrarily position two robots in a rectangular workspace. 
 Fig.~\ref{fig:shapeControlMathematica1} shows a Mathematica implementation of the algorithm, and is useful as a visual reference for the following description.

Assume two robots are initialized at $r_1$ and $r_2$ with corresponding goal destinations $g_1$ and $g_2$. 
The solution is a best-first search algorithm that maintains a list of possible paths (\emph{moves}), sorted according to an admissible heuristic on path length.
The algorithm works by expanding the path with the shortest estimated length.
Expanding a path means either moving directly to the goal, or pushing one robot to a wall and adjusting the relative position of the other robot.
As soon as the goal is reached, the algorithm returns this shortest path.

Denote the current positions of the robots  $r_1$ and $r_2$. 
Values $.x$ and $.y$ denote the $x$ and $y$ coordinates, i.e., $r_1.x$ and $r_1.y$ denote the $x$ and $y$ locations of $r_1$. 
The algorithm assigns a uniform control input at every instance.
The goal is to zero the error in both coordinates,
 $\Delta e =(\Delta e.x, \Delta e.y) =\Delta g - \Delta r = (g_2-g_1)- (r_2-r_1)$ using a shared control input. 
 The base case occurs when $\Delta e = (0,0)$. In this base case, the shortest path is a straight line between the current position of each robot to its goal position, as shown in Fig. \ref{fig:shapeControlMathematica1}a.  
If  $\Delta e\neq (0,0)$, we compute the two-move reachable sets generated by first immobilizing one particle by contacting the wall and then repositioning the second particle.
There are four reachable sets, as shown in Fig.~\ref{fig:TwoRegions}.
The horizontal (and vertical) reachable sets are equivalent in the  $\Delta$ configuration space, so we can plan in this space and choose to immobilize the particle closest to a wall. 
% When a particle contacts a wall, the other particle can move to any position in the reachable set while the first robot is immobile. 
%Two reachable sets are possible, horizontal and vertical. 
Each non-slip move changes the $\Delta$ configuration space, as shown in  Fig.~\ref{fig:regionMove}.
   

 %This algorithm exploits the position-dependent friction model \eqref{eq:frictionmodel}.
 %employing the assumption we have made earlier about the walls' friction. 
%Our algorithm uses an A*-like method to find the shortest path in each move. 
%If $(\Delta g.x, \Delta g.y)$ is in the reachable set, one robot touches a wall and the other robot zeros the error in one move. This is shown as $m_2$ in Fig. \ref{fig:reflection}. To find the best place to minimize $m_1$ and $m_3$, the touching robot's goal is reflected on that wall. 
%The minimum distance to get to the goal in two moves when the robot should touch the wall, is the straight line between the robot and the reflection of the goal position on that wall. 
If the goal configuration can be reached in three moves, then $m_1$  makes one particle hit a wall, $m_2$ adjusts the relative spacing error $\Delta e$ to zero, and  $m_3$ takes the particles to their final positions, as shown in Fig. \ref{fig:shapeControlMathematica1}b. 
$m_2$ cannot be shortened, so optimization depends on choosing the location where the robot hits the wall. 
 Since the shortest distance between two points is a straight line, reflecting the goal position across the boundary wall and plotting a straight line gives the optimal hit location, as shown in Fig. \ref{fig:reflection}.
That point is selected when possible, but if this point would cause $m_2$ to push the moving robot out of the workspace, the hit point is translated until the moving robot will not leave the workspace. If $m_2$ causes the two particles to overlap, we add or subtract $\epsilon$ to $m_2.x$ to avoid collisions. This is shown in Fig. ~\ref{fig:epsilon} with three different $\epsilon$ values.


\begin{figure}
\centering
\begin{overpic}[width=0.47\columnwidth]{twoRobotRegionH.pdf}\end{overpic}
\begin{overpic}[width=0.47\columnwidth]{twoRobotRegionV.pdf}\end{overpic}
\caption{\label{fig:TwoRegions}
Boundary interaction is used to change the relative positions of the robots. Each robot gets the same control input. 
(left) If robot 2 hits the bottom wall before robot 1 reaches a wall, robot 2 can reach anywhere along the green line, and  robot 1 can move to anywhere in the shaded area. 
(right) Similarly, if robot 2 hits the right wall before robot 1 reaches a wall, robot 2 can reach anywhere along the green line, and  robot 1 can move to anywhere in the shaded area. 
}
\end{figure}
If  $\Delta g$ is not in the reachable set, we choose the nearest reachable $\Delta x$ and $\Delta y$ to $\Delta g$. 
%For simplicity in this step, we choose the straight line from the touching robot to the wall for the first move. This may cause the algorithm not to return the shortest path, but the difference is not significant. 


Alg.~\ref{alg:optimalAlg} uses an admissible heuristic that adds the current path length to the greatest distance from each robot to their goal. This heuristic directs exploration by expanding favorable routes first.
\begin{align}\label{eq:admissibleHeuristic}
h(\text{\emph{moves}}, r_1,r_2,g_1,g_2) =& \sum_{i=1}^{|\text{\emph{moves}}|} \norm{\text{\emph{moves}}_i}  \\
&+  \max( \norm{g_1-r_1}, \norm{g_2-r_2} ) \nonumber
\end{align}


We  exploit symmetry in the solution by labeling the leftmost (or, if they have the same $x$ coordinate, the topmost) robot $r_1$. 
 If $r_1$ is not also the topmost robot, we mirror the coordinate frame about the right boundary. 
 As an example, consider the two starting positions, $r_1 =  (0.2, 0.2) $ and $r_2 = (0.8, 0.8)$. 
  Because the leftmost robot is not the topmost robot, we mirror the coordinate frame about the right boundary giving $r_1 = (0.2, 0.8)$ and $r_2 = (0.8,0.2)$. 
 After the path is found, we undo the mirroring to the output path. 
  Similarly, we exploit rotational symmetry and assume the command pushes a robot to hit the top wall.
   If a different wall is selected, we rotate the coordinate frame by 90$^{\circ}$, 180$^{\circ}$, or 270$^{\circ}$ counterclockwise and then push the robot to hit the top wall.  After the path is found, we undo the rotation. 
   This symmetry allows us to use a single function, Alg. \ref{alg:Wallup},  for collisions with all four walls. 

\begin{algorithm}[htb]
\caption{ { \sc 2-ParticlePathFinder}($r_1,r_2,g_1,g_2,L$)}\label{alg:optimalAlg}
\begin{algorithmic}[1]
%\scriptsize
\Require knowledge of current $(r_1,r_2)$ and goal $(g_1,g_2)$ positions of  two robots. 
$(0,0)$ is bottom corner,
 $L$ is length of the walls. 
 \emph{PathList} contains all the paths sorted by their path length plus an admissible heuristic. 
 \State  \emph{PathList} $\gets \{\}$
 \State $P \gets   \{ h(\{\},r_1,r_2,g_1,g_2  ) ,\{\},r_1,r_2\} $ \Comment $P$ contains the admissible heuristic, the move sequence, and the current robot positions
\While {$P.r_1 \ne g_1$ \textbf{and} $P.r_2 \ne g_2$}
\For{ $\theta \in \{0^\circ, 90^\circ, 180^\circ, 270^\circ \}$ }
\State $(r_1,r_2,g_1,g_2) \gets$ {\sc Rotate}($P.r_1,P.r_2,g_1,g_2,\theta$)
\State $\{d, $ \parbox[t]{.3\linewidth}{%
 \emph{moves,}$ r_1,r_2\} \gets$\\
 {\sc  PlanMoveUp}($r_1,r_2,g_1,g_2,L, P.\text{\emph{moves}}$)}
\State $(\text{\emph{moves}}, r_1,r_2) \gets$ {\sc Rotate}(\emph{moves}$, r_1,r_2,-\theta$)
\State   {\sc Push} $\{d, \text{\emph{moves}}, r_1,r_2\} $ onto \emph{PathList}
\EndFor
\State {\sc Sort}(\emph{PathList}) \Comment sort by admissible heuristic
\State $P \gets $ {\sc Pop} first element of \emph{PathList}
\EndWhile
\State \Return \emph{moves}
\end{algorithmic}
\end{algorithm}


\begin{algorithm}
\caption{{\sc PlanMoveUp}($r_1,r_2,g_1,g_2,L, $\emph{moves})}\label{alg:Wallup}
\begin{algorithmic}[1]
%\scriptsize
\Require knowledge of current $(r_1,r_2)$ and goal $(g_1,g_2)$ positions of  two robots. 
$(0,0)$ is bottom corner,
 $L$ is length of the walls. 
 The array \emph{moves} is the current sequence of moves up to the current position.
 Assume $r_1.x < r_2.x$ and $r_1.y \geq r_2.y$. If not, mirror the coordinate frame and swap the robots, then undo the mirroring before returning.
 $\epsilon $ is a small, nonzero, user-specified value.
 
\Ensure $(g_1, g_2) , (r_1, r_2)$ all at least $\epsilon$ distance from walls the goals and starting points have at least $\epsilon$ distance from each other.  
$m_1$ is the first move toward the wall or goal.
 $m_2$ is the second move adjusting $\Delta e$.
\State $\Delta e \gets (g_2 - g_1)- (r_2- r_1)$
\If {$\Delta e = (0,0)$} \Comment{base case}
\State $m_1  \gets g_2 -  r_2$
\State \emph{moves}$ \gets \{ \text{\emph{moves}},   m_1 \}$
\State $(r_1,r_2) \gets ${\sc ApplyMove}$(m_1,r_1,r_2)$
\State \Return $\{h(\text{\emph{moves}}, r_1,r_2,g_1,g_2), \text{\emph{moves}},r_1,r_2\}$
\EndIf
\If {$r_2.x - r_1.x - 1 + 2 \epsilon \leq  \Delta g.x \leq  1 ~\textbf{and}~r_2.y - r_1.y \leq \Delta g.y \leq  0$} \Comment{$\Delta g \in $ reachable region}
\State $m_1 \gets \left(\frac{1-r_1.y}{2-g_1.y-r_1.y} (g_1.x -r_1.x), 1-r_1.y \right)$
\If {$r_2.x + m_1.x >L$}
\State $m_1.x \gets 1- r_2.x$
%\EndIf
\Else \textbf{ if} {$r_2.x + m_1.x < 0$}  \textbf{then}
\State $m_1.x \gets -r_2.x$
\EndIf
\Else 
\State $m_1 = (0, 1-r_1.y)$
\State $\Delta g \gets $ closest reachable $(\Delta x,\Delta y)$.
%Adjust $\Delta g_x$ and $\Delta g_y$ %\Comment{To go to the nearest $\Delta g_x$ and $\Delta g_y$}
%\State $\Delta g_x = r_2.x - r_1.x - 1 + 2 \epsilon
\EndIf
\State \emph{moves}$ \gets \{ \text{\emph{moves}},   m_1\}$
\State $(r_1,r_2) \gets ${\sc ApplyMove}$(m_1,r_1,r_2)$
\State $m_2 \gets \Delta g -  (r_2- r_1)$
\If {robots on each other \textbf{or} on the wall}
\State Add $\pm \epsilon$ to $m_2.x$ to avoid collision
\EndIf
\State \emph{moves} $ \gets \{ \text{\emph{moves}},   m_2\}$
\State $(r_1,r_2) \gets ${\sc ApplyMove}$(m_2,r_1,r_2)$
\State \Return $\{h(\text{\emph{moves}}, r_1,r_2,g_1,g_2), \text{\emph{moves}},r_1,r_2\}$
\end{algorithmic}
\end{algorithm}








\begin{figure*}
\centering
\renewcommand{\figwid}{\columnwidth}
{\begin{overpic}[width =\figwid]{circleWallFrictionReachableRegions.pdf}
\end{overpic}
}\\

\caption{\label{fig:regionMove}{Workspace and $\Delta$ configuration space for three sets of robot configurations with the same final goal. The red square represents the starting $\Delta x$ and $\Delta y$ and the green circle represents the goal $\Delta x$ and $\Delta y$. 
The green rectangle illustrates reachable $\Delta x$ and $\Delta y$ when one particle is in contact with a horizontal wall and the blue rectangle illustrates the reachable region when in contact with a vertical wall.
}
\vspace{-1em}
}
\end{figure*}
\begin{figure}
\centering
\begin{overpic}[width=0.5\columnwidth]{Reflection.pdf}\end{overpic}
\caption{\label{fig:reflection}
If the goal configuration can be reached in three moves, the first move makes one particle hit a wall, the second move adjusts the relative spacing error $\Delta e$ to zero, and the third move takes the particles to their final positions. 
The second move cannot be shortened, so optimization depends on choosing the location where the robot hits the wall. 
 Since the shortest distance between two points is a straight line, reflecting the goal position across the boundary wall and plotting a straight line gives the optimal hit location.
} \vspace{-1em}
\end{figure}

\begin{figure*}
\centering
\renewcommand{\figwid}{0.5\columnwidth}
{\begin{overpic}[width =\figwid]{epsilonMin.pdf}%\put(20,20){$\epsilon = 0.001$}
\put(20, 3.1){{\scriptsize$\swarrow$}~$\epsilon = 0.001$}
\end{overpic}
\begin{overpic}[width =\figwid]{epsilonMed.pdf}%\put(20,20){$\epsilon = 0.05$}
\put(20, 3.9){{\tiny$\updownarrow$}~$\epsilon = 0.05$}
\end{overpic}
\begin{overpic}[width =\figwid]{epsilonMax.pdf}%\put(20,20){$\epsilon = 0.1$}
\put(20, 5){{\large$\updownarrow$}~$\epsilon = 0.1$}
\end{overpic}
}\\
\caption{\label{fig:epsilon}{Changing the minimum spacing $\epsilon$ changes the path.   $\epsilon$ is the minimum spacing between two robots and the minimum separation from the boundaries.}
\vspace{-1em}
}
\end{figure*}






\begin{figure*}
\centering
\renewcommand{\figwid}{2\columnwidth}
{\begin{overpic}[width =\figwid]{identical.png}
\end{overpic}

\vspace{1em}
\begin{overpic}[width =\figwid]{JustSimulationV6.png}
%\put(97.5,21){5}
%\put(97.5,17.25){4}
%\put(97.5,13.5){3}
%\put(97.5,24){6}
%\put(97.5,27){7}
%\put(97.5,10){2}
%\put(97.5,7.5){1}
%\put(97.5,5){0}
\end{overpic}
}\\
\caption{\label{fig:contourPlots}{Starting positions of robots $1$ and $2$ and goal position of robot $2$ are fixed, and $\epsilon=0.001$.
 The top row of contour plots show the distance if robot $1$'s goal position is varied in $x$ and $y$. The bottom row shows the number of moves required for the same configurations.}
\vspace{-1em}
}
\end{figure*}









