%%%%%%%%%%%%%%%%%%%%%%%%%%%%%%%%%%%%%%%%%%%%%%%%%%%%%%%%%%%
\section{Conclusion}\label{sec:conclusion}
%%%%%%%%%%%%%%%%%%%%%%%%%%%%%%%%%%%%%%%%%%%%%%%%%%%%%%%%%%%   
 
    %Micro- and nanorobotics have the potential to revolutionize many applications including targeted material delivery, assembly, and surgery.  
  The small size of micro and nano particles makes individual control and autonomy challenging, so currently these particles are steered by global control inputs such as magnetic fields or chemical gradients. To investigate this control challenge, this chapter introduced \href{http://www.swarmcontrol.net}{SwarmControl.net}, an open-source tool for large-scale user experiments where human users steer swarms of robots to accomplish tasks.  Analysis of the gameplay results revealed benefits of measuring and controlling statistics of the swarm rather than full state feedback, robustness to IID noise, and small effects of varying population size of large swarms.

Inspired by the three lessons from \href{www.swarmcontrol.net}{SwarmControl.net}, this paper designed controllers and controllability results using only the mean and variance of a particle swarm. 
We developed a hysteresis-based controller to regulate the position and variance of a swarm. We designed a controller for object manipulation using value iteration for path planning, regions for outlier rejection, and potential fields for minimizing moving the object backwards. 
All automatic controllers were implemented using 100 kilobots steered by the direction of a global light source.
These experiments culminated in an object manipulation task in a workspace with obstacles.
    
 % Manipulation by large populations of robots has many open questions. We invite other collaborators to submit their own experiments for large-scale trials to \href{http://www.swarmcontrol.net}{SwarmControl.net}.


Our goal in the next chapter is to apply force and torque to components and manipulating them through obstacles and each other. This work provides foundational algorithms and techniques for steering swarms, object manipulation, and addressing obstacle fields.%, but there are many opportunities to extend the work.

%Topics of interest include control with nonuniform flow such as fluid in an artery, gradient control fields like that of an MRI, competitive playing, multi-modal control, flexible workspaces, optimal-control, and targeted drug delivery in a vascular network.

%\begin{figure}
%\begin{overpic}[width = 0.48\columnwidth]{Worldbrowsing.pdf}\end{overpic}
%\begin{overpic}[width = 0.48\columnwidth]{USbrowsing.pdf}\end{overpic}
%\vspace{-1em}
%\caption{\label{fig:PlayerLocation}Demographic information on game player's location, provided by Google Analytics. Game players from 84 countries and 49 US states visited our site.
%\vspace{-2em}
%}
%\end{figure}

%\paragraph{Site modifications}
%  The current site is optimized for desktop and laptop users, and we currently do not support mobile users. Our IRB allows us to conduct demographic questionnaires, and we will implement these questionnaires in a future release--currently our only source of demographic data is Google Analytics.
%  
%  We are pursuing partnerships to increase the educational content on our website. Our goal is to highlight a variety of the leading micro- and nano-robotics labs and the challenges they are working on.



%\paragraph{Automatic controllers}
%We have compiled a large body of test results.  Our goal is to design automatic controllers using this data. One avenue is to identify the most proficient players and perform inverse optimal control algorithms to learn the cost functions used by the best players.  
%%  demographics questionaires: 


%We will use the video game described above as part of our outreach efforts.  The nature of ensemble control and the manipulation tasks will make this a puzzle game, where the user will need to determine the correct actions, and groups of actions, to accomplish the task.
%Discovering when a task changes from `fun' to `frustrating' is an active problem in game design.  Game companies (\emph{e.g.} ReignDesign Fig.~\ref{fig:Flockwork}) depend heavily on beta-testing to discover the point at which a task becomes frustrating to a user.  A model of what makes a task frustrating would revolutionize the game industry. Our estimate of this difficulty metric for massive manipulation is shown in Fig.~\ref{fig:GameDifficulty}.  ReignDesign has experience and current contracts designing educational games.  We can incorporate real-world physics to simulate robot control at the micro and the nano scale.
%
%Currently, we can use our control theoretic results to determine what tasks are possible. Unfortunately, this gives us no metric of human difficulty -- which tasks are easy for a human pilot.  What tasks should be off-loaded for computer control?  Fortunately \emph{gamification} provides built-in tools to gather this data in a transparent manner with the user's consent by measuring the time and number of actions required to complete each task, and through \emph{leaderboards} \cite{Zichermann2011,Kapp2012}, which rank users based on the efficiency of their solutions.  Key to the success of this endeavor is an engaging story.  We have the genesis of the a game in \cite{Becker2012l}, with game play based on the desire to steer many robots equipped with suction-cup darts in a surprise attack against an older sister. Our hope is to use current micro- and nanorobotics research to create an engaging story users will delight to immerse themselves in.

%In order to understand the difficulty of the tasks, we can measure the time and number of actions required to complete each task, as in Fig.~\ref{fig:Flockwork}.  This information will be collected to maintain a `top scores' list on the website, and help us answer questions about task difficulty.

%hopefully a discussion about flockworks, a successful multi-agent simulation (but actually a mobile app) here....

%http://gamification.org/wiki/Game_Features/Leaderboards

%   - Chemistry, Micro/nano education, molecular dynamics
