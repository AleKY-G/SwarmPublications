
%%%%%%%%%%%%%%%%%%%%%%%%%%%%%%%%%%%%%%%%%%%%%%%%%%%%%%%%%%%
\section{Related Work}\label{sec:RelatedWork}
%%%%%%%%%%%%%%%%%%%%%%%%%%%%%%%%%%%%%%%%%%%%%%%%%%%%%%%%%%%


%Unlike \emph{caging} manipulation, where robots form a rigid arrangement around an object~\cite{Sudsang2002,Fink2007}, our swarm of robots is unable to grasp the blocks they push, and so our manipulation strategies are similar to \emph{nonprehensile manipulation} techniques, e.g.~\cite{Lynch1999}, where forces must be applied along the center of mass of the moveable object. 


Robotic manipulation by pushing has a long and successful history\cite{Lynch1999,Lynch1996,Akella2000,Bernheisel2006}.  Key developments introduced the notion of stable pushes and a friction cone.  A \emph{stable push} is a pushing operation by a robot with a flat-plate pushing element in which the object does not change orientation relative to the pushing robot\cite{Lynch1999}.  The \emph{friction cone} is the set of vector directions a robot in contact with an object can push that object with a stable push.
Stable pushes can be used as primitives in an rapidly-expanding random tree to form motion plans.
A key difference is that our robots are compliant and tend to flow around the object, making this similar to fluidic trapping~\cite{Armani2006,Becker2009}.  
%\subsection{Block-pushing and Compliant Manipulation}

While ferrous particles tend to clump in a magnetic field, the magnetotactic bacteria of~\cite{martel2015magnetotactic,ou2012motion} are directed by the orientation of the magnetic field and do not suffer from magnetic aggregation.

Controlling the \emph{shape}, or relative positions, of a swarm of robots is a key ability for a myriad of applications.  Correspondingly, it has been studied from a control-theoretic perspective in  both centralized, e.g. virtual leaders in \cite{egerstedt2001formation}, and decentralized approaches, e.g. decentralized control-Lyapunov function controllers in~\cite{hsieh2008decentralized}. Most approaches assume a level of intelligence and autonomy in the individual robots that exceeds the capabilities of current micro- and nano-robots~\cite{martel2015magnetotactic,Xiaohui2015magnetiteMicroswimmers}.
Instead, this chapter focuses on centralized techniques that apply the same control input to each member of the swarm, as in~\cite{Becker2013b}.


