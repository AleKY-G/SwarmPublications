%%%%%%%%%%%%%%%%%%%%%%%%%%%%%%%%%%%%%%%%%%%%%%%%%%%%%%%%%%%
\section{Conclusion}\label{sec:conclusion}
%%%%%%%%%%%%%%%%%%%%%%%%%%%%%%%%%%%%%%%%%%%%%%%%%%%%%%%%%%%


This chapter presented techniques for controlling the orientation of an object by manipulating it using a swarm of simple robots with global inputs.
The chapter provided algorithms for precise orientation control, as well as demonstrations of orientation control. 


%Future efforts should be directed toward optimizing torque control, examining the effects of Brownian noise, applying the techniques to hardware robots, pose control for multiple part assembly, and manipulation in a crowded workspace.
The control laws in this chapter used only the mean and variance of the swarm.  The control techniques may be optimized using high-order moments, or by stochastic modeling of the collisions between swarm members and the object.

