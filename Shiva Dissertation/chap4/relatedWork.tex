
%%%%%%%%%%%%%%%%%%%%%%%%%%%%%%%%%%%%%%%%%%%%%%%%%%%%%%%%%%%
\section{Related work}\label{sec:RelatedWork}
%%%%%%%%%%%%%%%%%%%%%%%%%%%%%%%%%%%%%%%%%%%%%%%%%%%%%%%%%%%

Controlling the \emph{shape}, or relative positions, of a swarm of robots is a key ability for a range of applications.  Correspondingly, it has been studied from a control-theoretic perspective in  both centralized and decentralized approaches. For examples of each, see the centralized virtual leaders in \cite{egerstedt2001formation}, and the  gradient-based decentralized controllers  using control-Lyapunov functions in~\cite{hsieh2008decentralized}. However, these approaches assume a level of intelligence and autonomy in individual robots that exceeds the capabilities of many systems, including current micro- and nano-robots.  Current micro- and nano-robots, such as those in~\cite{Chowdhury2015,martel2015magnetotactic,Xiaohui2015magnetiteMicroswimmers} lack onboard computation.

This chapter focuses on centralized techniques that apply the same control input to both particles. 
Precision control requires breaking the symmetry caused by the uniform input.  
Symmetry can be broken using particles that respond differently to the uniform control signal, either through agent-agent reactions \cite{bertozzi2015ring}, or engineered inhomogeneity  \cite{Donald2013,bretl2007,beckerIJRR2014}. 
 The magnetic gradients of MRI scanners are \emph{uniform}, meaning the same force is applied everywhere in the workspace\cite{nosrati2018development}.
 This work assumes a uniform control with homogenous particles, as in~\cite{AaronManipulation2013}, and breaks the control symmetry using obstacles in the workspace. 

%The techniques in this paper are inspired by artificial force-fields. 

%\emph{Fluid-flow:} 
%Fluid flow along boundaries generates a shear force that pushes different parts of a body in opposing directions. 
%Most introductory fluid dynamics textbooks provide models~\citep{Munson2013}.
%Similarly, a swarm of robots under global control pushed along a boundary will experience shear forces.  
%This is a position-dependent force, and so can be exploited to control the configuration or shape of the swarm.  
% \citep{spears2006physics}~used these forces to disperse a swarm's spatial position for coverage for physics-based swarm simulations.

%\emph{Artificial Force-fields:}
%Much research has focused on generating non-uniform artificial force-fields that can be used to rearrange passive components. 
Alternative techniques rely on non-uniform inputs, such as artificial force-fields.
Applications have included techniques to design shear forces for sensorless manipulation of a single object by~\cite{lamiraux+2001:ra}.  
\cite{vose2012sliding} demonstrated a collection of 2D force fields generated by six degree-of-freedom vibration inputs to a rigid plate.  These force fields, including shear forces, could be used as a set of primitives for motion control to steer the formation of multiple objects. %However unlike the uniform control model in this paper, their control was multi-modal and position-dependent.
%\todo{talk about obstacles, think about adding goldberg}

%This paper develops control algorithms using uniform control fields, such as the magnetic resonance navigation \cite{nosrati2018development}.%field in a clinical MRI [insert a recent reference from Sylvain Martel using MRI].
Similarly, much recent work in magnet control has focused on exploiting inhomogeneities in the magnetic field to control multiple micro particles  using gradient-based pulling~\cite{Salmanipour2018EightDOF,Denasi2018independent}.  
Unfortunately, using large-scale external magnetic fields makes it challenging to independently control more than one microrobot unless the  distance between the electromagnetic coils is at the same length scales as the robot workspace~\cite{diller2016six, Denasi2018independent, Salmanipour2018EightDOF }. In contrast, % to methods that exploit inhomogeneities in the magnetic field to control multiple micro particles, e.g. \cite{Denasi2018independent}, that exploited nonlinearities generated by four magnetic coils in close proximity to the workspace to achieve trajectory control of two microspheres, 
 this chapter requires only a controllable constant gradient in orthogonal directions to position the particles.
% Systems like this one are poorly suited for PRM and RRT*-type methods\cite{lavalle2006planning} because if during a movement a collision occurs, that movement is irreversible.
   %Flow in a pipe or body lume

If a control input causes the particles to collide with obstacles at different times, inverting the control input does not undo the action. 
 Due to this lack of time-reversibility, techniques that require a bidirectional graph, e.g. PRM \cite{kavraki1996probabilistic} and RRT* \cite{lavalle2006planning} are not suitable.
  Instead, this chapter employs a greedy search strategy. 