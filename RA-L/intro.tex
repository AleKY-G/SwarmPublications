\section{Introduction}\label{sec:Intro}
Particle swarms steered by a global force are common in applied mathematics, biology, and computer graphics. 
\begin{align}
[\dot{x}_i, \dot{y}_i]^\top = [u_x, u_y]^\top, \qquad i \in [1,n] \label{eq:swarmDynamics}
\end{align}
The control problem is to design $u_x(t), u_y(t)$ to make all $n$ particles achieve a task.
As a current example, micro- and nano-robots can be manufactured in large numbers, see~\citep{Chowdhury2015}, \citep{martel2014computer}, \citep{kim2015imparting}, \citep{Donald2013}, \citep{Ghosh2009}, \citep{Ou2013} or \citep{qiu2015magnetic}.
Someday large swarms of robots will be remotely guided
  ex vivo to assemble structures in parallel and 
 through the human body to cure disease, heal tissue, and prevent infection. %Both tasks are high   as described in \citep{munoz2014review} \citep{sitti2015biomedical} . 
 For each application, large numbers of micro robots are required to deliver sufficient payloads, but the small size of these robots makes it difficult to perform onboard computation.  Instead, these robots are often controlled by a global, broadcast signal. 
These applications require techniques to shape the swarm that can reliably exploit large populations despite high under-actuation.  
 

%\begin{figure}
%\centering
%\begin{overpic}[width=0.9\columnwidth]{Leaf.jpg}\end{overpic}
%\caption{\label{fig:vascularNetwork}
%Vascular networks are common in biology such as the circulatory system and cerebrospinal spaces, as well as in porous media including sponges and pumice stone.  Navigating a swarm using global inputs, where each member receives the same control inputs, is challenging
%due to the many obstacles. 
%This paper focuses on using boundary walls and wall friction to break the symmetry caused by the global input and control the shape of a swarm.} 
%\end{figure}

\begin{figure}
\centering
\begin{overpic}[width=0.95\columnwidth]{IntroSwarmPic.pdf}\end{overpic}
\caption{\label{fig:IntroPic}
Swarm of kilobots programmed to move toward the brightest light source as explained in \S \ref{sec:expResults}. The current covariance ellipse and mean are shown in red, the desired covariance is shown in green.  Navigating a swarm using global inputs is challenging because each member receives the same control inputs. 
This paper focuses on using boundary walls and wall friction to break the symmetry caused by the global input and control the shape of a swarm.} 
\end{figure}


%One surprising result was that humans that only knew the swarm's mean and covariance completed the task faster that humans who knew the position of every robot~\citep{Becker2013b}. Our previous work focused on a block-pushing task, where a swarm of robots pushed a larger block through a 2D maze. 
Even without obstacles or boundaries, the mean position of the swarm in \eqref{eq:swarmDynamics} is controllable.  By adding rectangular boundary walls, some higher-order moments, such as the swarm's position variance orthogonal to the boundary walls ($\sigma_x$ and $\sigma_y$ for a workspace with axis-aligned walls), are also controllable~\citep{ShahrokhiIROS2015}. 
A limitation is that global control can only compress a swarm orthogonal to obstacles, but navigating through narrow passages often requires control of the variance and the covariance of the swarm.


The paper is arranged as follows. First \S\ref{subsec:FluidInTank} provides analytical position control results in two canonical workspaces with frictionless walls.  These results are limited in the set of shapes that can be generated.  To extend the range of possible shapes, \S \ref{subsec:WallFriction} introduces wall friction to the system model.  We prove that two orthogonal boundaries with high friction are sufficient to arbitrarily position two robots in \S \ref{sec:PostionControl2Robots}, and \S \ref{sec:PostionControlnRobots} extends this to prove a rectangular workspace with high-friction boundaries can position a swarm of $n$ robots arbitrarily within a subset of the workspace.
\S \ref{sec:simulation} describes implementations of both position control algorithms in simulation and  \S \ref{sec:expResults} describes experiments with a hardware setup and up to 100 robots, as shown in Fig.~\ref{fig:IntroPic}. After a review of recent related work in \S \ref{sec:RelatedWork}, we end by suggesting directions for future research in \S \ref{sec:conclusion}.
%For controlling $\sigma_{xy}$, we prove that the swarm position covariance $\sigma_{xy}$ is controllable given boundaries with non-zero friction. 
%We then prove that two orthogonal boundaries with high friction are sufficient to arbitrarily position a swarm of $n$ robots. 
%We conclude by designing controllers that efficiently regulate $\sigma_{xy}$.
%This paper
%(1) proves that the swarm position covariance $\sigma_{xy}$ is controllable given boundaries with non-zero friction,  %where do we do this?
%(2) proves that two orthogonal boundaries with high friction are sufficient to arbitrarily position two robots, 
%(3) proves that two orthogonal boundaries with high friction are sufficient to arbitrarily position a swarm of $n$ robots, 
%(4) shows full-state position control with 2 or more robots using  extensive simulations, and
%(5) demonstrate covariance control on our hardware platform with a large number of hardware robots.
%TODO JOURNAL: design controllers that efficiently regulate $\sigma_{xy}$.
%TODO JOURNAL: We will design Lyapunov-inspired controllers for $\sigma_{xy}$ to prove controllability. 
%TODO JOURNAL:  and rank controllability as a function of friction.
% TODO: JOURNAL: and vary wall friction by laser-cutting boundary walls with a variety of profiles. 
%\begin{figure}[t]
%\centering
%\begin{overpic}[width = \columnwidth]{Covariance.jpg}\end{overpic}
%\vspace{-1em}
%\caption{\label{fig:covFriction} Maintaining group cohesion while steering a swarm through an arbitrary maze requires covariance control.
%}\vspace{-1em}
%\end{figure}
% Our paper is organized as follows.  After a discussion of related work in Section \ref{sec:RelatedWork}, we describe our experimental methods for an online human-user experiment in Section \ref{sec:expMethods}.  We report the results of our experiments in Section \ref{sec:expResults}, discuss the lessons learned in Section \ref{sec:discussion}, and end with concluding remarks in Section \ref{sec:conclusion}.